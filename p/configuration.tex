% \pagestyle{giruled}
% \pagestyle{scrheadings}
\setlist{nosep}
\sloppy
\iftutex\newfontfamily\DejaSans{DejaVu Sans}\fi
\relpenalty=10000
\binoppenalty=10000
% \geometry{a4paper,top=1cm,bottom=2cm,left=.5cm,right=.5cm}
%\renewcommand{\baselinestretch}{1.3}

% mathtools option to transform := into \colonequals
% \mathtoolsset{centercolon}

% From      https://tex.stackexchange.com/a/526403

\makeatletter
\xpatchcmd \thmt@restatable{\@xa\protected@edef\csname the#2\endcsname}
                           {\def\thmt@tmp@restatename{#3}\@xa\protected@edef\csname the#2\endcsname}{}{\fail}


\renewtheoremstyle{plain}%
  {\item[
   \ifcsname 
        thmt@tmp@restatename\endcsname 
        \ifthmt@thisistheone 
            \hskip\labelsep \theorem@headerfont ##1\ ##2
        \else
            \mbox{\hyperref[thmt@@\thmt@tmp@restatename]{\theorem@headerfont ##1~##2}} 
        \fi
   \else 
        \hskip\labelsep \theorem@headerfont ##1\ ##2
    \fi
   \theorem@separator]}%
  {\item[
   \ifcsname 
        thmt@tmp@restatename\endcsname
        \ifthmt@thisistheone 
            \hskip\labelsep \theorem@headerfont ##1\ ##2
        \else
            \mbox{\hyperref[thmt@@\thmt@tmp@restatename]{\theorem@headerfont ##1~##2}}
        \fi
   \else 
        \hskip\labelsep \theorem@headerfont ##1\ ##2
   \fi
   \  (##3)\theorem@separator]%
  }

\makeatother


\numberwithin{equation}{section}
\newcommand{\creflastconjunction}{ e~}
\newcommand{\creflastgroupconjunction}{ e~}
\crefname{equation}{equação}{equações}
% \declaretheorem[name=Teorema,numberwithin=section]{teorema}
\newtheorem{teorema}[equation]{Teorema}
\crefname{teorema}{teorema}{teoremas}
\newtheorem{lema}[equation]{Lema}
\crefname{lema}{lema}{lemas}
\newtheorem{cor}[equation]{Corolário}
\crefname{cor}{corolário}{corolários}
\newtheorem{prop}[equation]{Proposição}
\crefname{prop}{proposição}{proposições}
\newtheorem{defin}[equation]{Definição}
\crefname{defin}{definição}{definições}
\newtheorem{exem}[equation]{Exemplo}
\crefname{exem}{exemplo}{exemplos}
\newtheorem{exer}[equation]{Exercício}
\crefname{exer}{exercício}{exercícios}
\newtheorem{conjectura}[equation]{Conjectura}
\crefname{conjectura}{conjectura}{conjecturas}
\newtheorem{question}[equation]{Pergunta}
\crefname{question}{pergunta}{perguntas}
\newtheorem{remark}[equation]{Observação}
\crefname{remark}{observação}{observações}
\newtheorem{problema}[equation]{Problema}
%\numberwithin{problema}{subsection}
\crefname{problema}{problema}{problemas}

\newcounter{prob}\newenvironment{prob}{\refstepcounter{prob}\textbf{\theprob.}\rmfamily}{}
\crefname{prob}{problema}{problemas}

\newenvironment{nalign}{
    \begin{equation}
    \begin{aligned}
}{
    \end{aligned}
    \end{equation}
    \ignorespacesafterend
}

\newenvironment{eqs}{
    \begin{equation}
    \begin{split}
}{
    \end{split}
    \end{equation}
    \ignorespacesafterend
}


