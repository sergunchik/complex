% usar \defcommand em vez de \newcommand ou \renewcommand
% funciona como \def em tex - vai definir, se ainda não definida,
% se já definida, vai redefinir
\makeatletter\def\defcommand{\@ifstar\defcommand@S\defcommand@N} \def\defcommand@S#1{\let#1\outer\renewcommand*#1} \def\defcommand@N#1{\let#1\outer\renewcommand#1} \makeatother

\mathchardef\mhyphen="2D
\newcommand\dash{\nobreakdash-\hspace{0pt}}

\providecommand{\tightlist}{%
  \setlength{\itemsep}{0pt}\setlength{\parskip}{0pt}}

% mathcal e mathb letras
\def \CC {\mathcal{C}}  % 

%mathbf letras

\def \A {\mathbf{A}}    % anel
\def \C {\mathbf{C}}    % números complexos
\def \Ch {\widehat{\C}} % esfera de Riemann
\def \D {\mathbf{D}}    % disco
\def \F {\mathbf{F}}    % corpos finitos F_q
\def \H {\mathbf{H}}	% semiplano superior
\def \P {\mathbf{P}}    % números primos
\def \Q {\mathbf{Q}}    % números rationais
\def \R {\mathbf{R}}    % números reais
\def \Z {\mathbf{Z}}    % números inteiros

\def \M  {\mathcal{M}}    % meromorphic functions
\def \O  {\mathcal{O}}    % holomorphic functions

\newcommand\cat{\iftutex\DejaSans\text{😺}\else\mathcal{G}\fi}

\newcommand{\lra}{\longrightarrow}
\newcommand\ii{\ensuremath{\mathrm{i}}}  % raiz quadrática de -1
\newcommand\dpi{2\pi\ii}                 % 2πi = ∫ dz/z
\newcommand\idpi{\frac{1}{\dpi}}         % 1/(2πi)
\newcommand\ol{\overline}

\DeclareMathOperator{\dlog}{dlog}
\DeclareMathOperator{\res}{res}
\DeclareMathOperator{\sinal}{sinal}
\renewcommand{\ord}{ord}
\DeclareMathOperator{\mdc}{mdc}
\DeclareMathOperator{\mmc}{mmc}
\DeclareMathOperator{\codim}{codim}
\DeclareMathOperator{\Hom}{Hom}
\DeclareMathOperator{\End}{End}
%\defcommand{\Im}{\mathrm{Im} }
\let\Im\relax
\DeclareMathOperator{\Im}{Im}
\DeclareMathOperator{\Ker}{Núc}
\DeclareMathOperator{\Coker}{Conúc}
\DeclareMathOperator{\Aut}{Aut}
\DeclareMathOperator{\Tr}{Tr}
\DeclareMathOperator{\tr}{tr}
%\DeclareMathOperator{\dim}{dim}
\DeclareMathOperator{\PGL}{PGL}
\DeclareMathOperator{\GL}{GL}

\defcommand{\mat}[1]{\big(\begin{smallmatrix} #1 \end{smallmatrix}\big)}

\providecommand{\arxiv}[1]{\href{http://arxiv.org/abs/#1}{arXiv:#1}}
\providecommand{\doi}[1]{\href{http://dx.doi.org/#1}{\texttt{doi:#1}}}

