\usepackage{iftex}                      % check which engine is compiling this file
\iftutex                                % xelatex or lualatex specific first
 \usepackage{fontspec}                  % font selection in xelatex/lualatex
 \defaultfontfeatures {Ligatures=TeX}
 \usepackage{unicode-math}              % amssymb+amsmath equiv., to show $θ$ as θ
 \unimathsetup{math-style=TeX}
 \setmathfont{Stix Two Math}
 \ifLuaTeX
  % \usepackage{luatex85}               % compatibility of luatex with xy
   \usepackage{lualatex-math}
 \fi
\else                                   % 8bit (e.g. pdflatex) specific
\usepackage[T1]{fontenc}                % use 8-bit T1 fonts
% \usepackage[utf8]{inputenc}           % allow utf-8 input, for texlive-2016 and older
\fi
\usepackage{alphabeta}                  % type αβγ directly, in LuaLaTeX use {unicode-math} (above)

\usepackage{babel}
%\usepackage{polyglossia}
%\setdefaultlanguage{portuguese}

\usepackage[babel]{microtype}
%\usepackage{libertine}
%\usepackage[default]{frcursive}
%\usepackage[eulergreek,noplusnominus,noequal,nohbar,nolessnomore,noasterisk]{mathastext}

\usepackage[dvipsnames]{xcolor}
\usepackage[fleqn,leqno]{mathtools}
\usepackage[fleqn,leqno]{amsmath}
\usepackage[amsthm]{ntheorem}
\usepackage{thmtools,xpatch}
\usepackage{thm-restate}
%\usepackage{amsthm}
\usepackage{hyperref}
\hypersetup{
 unicode,
% pagebackref=true,
 hypertexnames = false,
 bookmarksdepth = 2, bookmarksopen = true, bookmarksnumbered, colorlinks,
 linkcolor={red!50!black},
 citecolor={blue!50!black},
 urlcolor={blue!80!black},
 pdfstartview={XYZ null null 1},
}
\usepackage[nameinlink,capitalize,noabbrev]{cleveref}
\usepackage[ocgcolorlinks]{ocgx2}[2017/03/30]   % linebreak links and uncolor in print

\usepackage[mark]{gitinfo2}
%\usepackage{graphicx}
%\usepackage{longtable}
%\usepackage{geometry}
%\usepackage{ulem} % sout
