\pnove*
\begin{proof}[\Cref{p9}]

Uma solução óbvia é $f(z) = \sin(z)$ e $g(z) =\cos(z)$.
Também as equações tem uma simetria no domínio:
se $f(z),g(z)$ é uma solução, então $f_k(z) = f(kz), g_k(z) = g(kz)$
também é uma solução para qualquer $k\in\C$.
Fórmula trigonométrica do seno da soma
\[ \sin(z+w) = \sin(z) \cos(w) + \cos(z) \sin(w) \]
tem uma versão para cosseno:
\[ \cos(z+w) = \cos(z)\cos(w) - \sin(z)\sin(w). \]
A priori a gente não sabe $g(z+w)$, mas podemos derivar o valor dele
da equação $g(z+w)^2 = 1 - f(z+w)^2$ substituindo $f(z+w)$
por $f(z)g(w)+g(z)f(w)$:
\begin{align*}
g(z+w)^2 
\\ &= 1\cdot 1 - f(z+w)^2
\\ &= \big(f(z)^2+g(z)^2\big)\cdot\big(f(w)^2+g(w)^2\big) - \big(f(z+w)\big)^2
\\ &= \big(f(z)^2+g(z)^2\big)\cdot\big(f(w)^2+g(w)^2\big) - \big(f(z)g(w)+g(z)f(w)\big)^2
\\ &= \big(g(z)g(w)-f(z)f(w)\big)^2,
\end{align*}
então para todos $z,w\in\C$ temos
\[ (g(z+w)-g(z)g(w)+f(z)f(w)) \cdot (g(z+w) + g(z)g(w)-f(z)f(w)) = 0. \]
I.e. para todos $z,w\in\C$ ou
\[ V(z,w) := g(z+w) - \big(g(z)g(w) - f(z)f(w)\big) = 0 \]
ou 
\[ F(z,w) := g(z+w) + \big(g(z)g(w)-f(z)f(w))\big) = 0. \]
Queremos ver que é $V(z,w)=0$.

\end{proof}


