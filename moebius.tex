\subsection{Transformadas de Möbius}

\begin{defin}
Uma \emph{transformação de Möbius} é uma bijeção da esfera de Riemann $\Ch$ dada por
$z\mapsto \frac{az+b}{cz+d}$ com $a,b,c,d\in\C$ e $ad - bc \neq 0$ .
\end{defin}

\begin{problema}
Provar que
\begin{enumerate}
\item O grupo das transformações de Möbius é gerado por $z\mapsto \lambda z$,
$z\mapsto z+1$ e $z\mapsto 1/z$.
\item As transformações de Möbius mandem os círculos generalizados (círculos ou retas)
nos círculos generalizados.
\item As transformações de Möbius respeitam as reflexões:
as imagens de pontos simétricos com respeito de um círculo são simétricos com respeito à imagem dele.
\item As transformações de Möbius são conformes, i.e. preservam os ângulos.
\end{enumerate}
\end{problema}

\begin{problema}
Provar que um par de pontos $z_1,z_2\in \Ch$ são simétricos com respeito de um círculo $C$
se e somente se qualquer círculo que passa de $z_1$ e $z_2$ intersecta o círculo $C$
nos ângulos retos.
\end{problema}

\begin{problema}
Provar que por qualquer par de círculos em $\Ch$ existe uma transformada de Möbius que os manda no
um dos seguintes
\begin{enumerate}
\item par de círculos concêntricos,
\item ou par de retas paralelas,
\item ou par de retas passando de zero.
\end{enumerate}
\end{problema}

\begin{problema}
Sejam $C$ um círculo com centro $O$ e $C' = \phi(C)$ a imagem dela,
o círculo com centro $O'$. Dar um exemplo em qual $O' \neq \phi(O)$.
\end{problema}

\begin{problema}[Círculos de Apolônio]
Sejam $z_1,z_2\in\C$. Provar que por cada $K>0$ o lugar de pontos
$S_K = \{z \in \C: \frac{|z-z_1|}{|z-z_2|} = K \}$ é um círculo, chamado um círculo de Apolônio.
Provar que qualquer círculo $S_K$ interseta qualquer círculo que passa de $z_1$ e $z_2$ em ângulos retos.
Desenha essas duas famílias de círculos.
\end{problema}

\begin{problema}
Sejam $C$ e $D$ um par de círculos disjuntos. Suponha que há $N$ círculos
$S_1,\dots,S_N$ tais que cada $S_i$ e tangente a $C$ e a $D$,
e também $S_i$ é tangente a $S_{i+1}$ e $S_N$ é tangente a $S_1$ (pode desenhar-las em sentido anti-horário).
Provar que para qualquer outros $N$ círculos $S_1',\dots,S_N'$ tais que
cada $S_i'$ é tangente aos $C$ e $D$ e também $S_1'$ é tangente ao $S_2'$,\dots,$S_{N-1}'$ é tangente ao $S_N'$
temos necessariamente que $S_N'$ é tangente ao $S_1'$.
\end{problema}

\begin{problema}
Sejam 
$z_0,z_1,z_\infty \in\Ch$ tais que $z_0\neq z_1$, $z_0\neq z_\infty$, $z_1\neq z_\infty$
e 
$w_0,w_1,w_\infty \in\Ch$ tais que $w_0\neq w_1$, $w_0\neq w_\infty$, $w_1\neq w_\infty$.
Provar que existe a única transformada de Möbius $\phi : \Ch \to \Ch$ tal que
$\phi(z_0) = w_0$, $\phi(z_1) = w_1$ e $\phi(z_\infty) = w_\infty$.
\end{problema}

\begin{problema}
Desenhe as imagens das retas verticais $\Re z = x$ e horizontais $\Im z = y$
com respeito de uma transformada de Cayley $w = \frac{z-i}{z+i}$.
\end{problema}

\begin{problema}
Desenhe as imagens dos círculos que passam de pontos $z=i$ e $z=-i$ com respeito da transformada
de Cayley $w = \frac{z-i}{z+i}$.
\end{problema}

\begin{problema}
Ache $\phi(D)$ para $D$ e $w=\phi(z)$ dado por
\begin{enumerate}
\item $D = \{z\in\C: \Im(z) = 1\}$, $\phi(z) = \frac{z-1}{z+1}$.
\item $D = \{z\in\C: |z-1/2| = 1/4\}$, $w = -1/z$.
\item $D = \{z\in\C: |z|\geq 1 \}$, $w = 2 \frac{2z-1}{z-2}$.
\end{enumerate}
\end{problema}
