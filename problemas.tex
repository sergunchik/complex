\subsection*{Notações.}
As regiões comuns:
$\Ch = \C \bigcup \infty$ --- a \emph{esfera de Riemann},
$H = \{z\in\C: \Im(z)>0\}$ --- o \emph{plano superior},
$D_R = \{z\in\C: |z|<R\}$ --- o \emph{disco} de raio $R>0$,
$\D$ --- o \emph{disco unitário},
$\D^* = \D - \{0\}$ --- o \emph{disco perfurado},
$\A_R := \{z\in\C: 1/R < |z| < R \}$ o \emph{anel} para $R\in(1,+\infty)$,
$D\subset \Ch$ --- uma região;
$\O(D)$ --- o conjunto das funções holomorfas na $D$;
$\phi : \Ch\to\Ch$ --- as \emph{transformações de Möbius} 
$\phi(z) = \frac{az+b}{cz+d}$
com $a,b,c,d\in\C$ e $ad\neq bc$.

\subsection{Outubro 18}

\begin{restatable}{prob}{p1}\label{p1} % [1]
Para $x\in(-1,1)$
construir um biholomorfismo %elementar entre
$\psi: \H \to \D - [x,1)$.
% o plano superior
% e o complemento para o intervalo $[x,1) \subset \R$
% no disco unitário.
\end{restatable}

\begin{restatable}{prob}{p2}\label{p2} % [2]
Para círculo generalizado (círculo ou reta) $S \subset \Ch$ define
uma \emph{refleção}
$R_S : \Ch \to \Ch$ como a refleção em reta ou a inversão em círculo.
% Seja $\phi: \Ch \to \Ch$ uma transformação de Möbius, i.e. $\phi(z) = \frac{az+b}{cz+d}$
%com $ad\neq bc$.
Provar que as transformações de Möbius $\phi : \Ch\to\Ch$
preservam as refleções, i.e. para qualquer
círculo $S$ e a imágem dela $\phi(S)$ temos $\phi(R_S z) = R_{\phi(S)} \phi(z)$.
\end{restatable}

\begin{restatable}{prob}{p3}\label{p3} % [3]
Sejam $C_1,C_2 \subset \Ch$ um par de círculos disjuntos, i.e. $C_1 \bigcap C_2 = \emptyset$,
e $D \subset \Ch$ uma região com a fronteira $\partial D = C_2 - C_1$.
(a) Provar que existe uma transformação de Möbius $\phi$ tal que
$\phi(D) = \A_R$ para algum $R>1$. %  = \{ z\in\C: 1/R < |z| < R \}$.
(b) Qual é o lugar geométrico de círculos $S$, que tocam ambos $C_1$ e $C_2$?
\end{restatable}

\begin{restatable}{prob}{p4}\label{p4} % [4]
Define $Sf = \frac{f'''}{f'} - \frac{3}{2} \big(\frac{f''}{f'}\big)^2$.
Provar para qualquer funções $f,g$ com singularidades isoladas e qualquer
transformação de Möbius $\phi$ que
(a) $S \phi = 0$,
(b) $S (\phi \circ f) = S f$,
(c) $S(f \circ \phi) = \big((Sf)\circ\phi\big) \cdot (\phi')^2$,
(d) $S(f\circ g) = \big((Sf)\circ g\big) \cdot (g')^2 + Sg$.
\end{restatable}

\begin{restatable}{prob}{p5}\label{p5} % [5]
Achar $\lim_{\rho\to+0} \int_\alpha^\beta f(\rho\cdot e^{i\phi}) d\phi$
para $f : D_\epsilon\to\C$ contínua.
\end{restatable}

\begin{restatable}{prob}{p6}\label{p6} % [6]
Seja $\omega_t = \frac{e^{itz} dz}{(z+z^{-1})^3}$.
(a) Classificar singularidades de $\omega_t$ na $\Ch$
para cada $t\in\C$.
(b) Calcular $I(t) = \int_\R \omega_t$.
\end{restatable}

\begin{restatable}{prob}{p7}\label{p7} % [7]
Para $f\in\O(\D^*)$ % uma função holomorfa no disco perfurado $D^*$. % = P(0,1) = \{z\in\C: 0<|z|<1\}$.
suponha que $\frac{f'(z)}{f(z)}$
tem um polo simples em $0$. Provar que
(a) a função $f$ é meromorfa em $0$,
(b) e $f$ tem zero ou polo em $0$.
\end{restatable}

\begin{restatable}{prob}{p8}\label{p8} % [8]
Seja $f(z) = \frac{P(z)}{Q(z)}$ para um par de polinómios $P,Q\in\C[z]-\{0\}$.
Suponha que para qualquer $|z|=1$ temos $|f(z)|=1$. Classificar $f$,
i.e. dar uma condição suficiente e necessária sobre os polos e nulos de $f$.
\end{restatable}

\begin{restatable}{prob}{pnove}\label{p9} % [9]
%Quais podem ser as funções inteiras 
Classificar $f,g\in\O(\C)$ tais que
$f^2+g^2=1$ e $f(z+w) = f(z) g(w) + f(w) g(z)$.
\end{restatable}

\begin{restatable}{prob}{p10}\label{p10} % [10]
Suponha que $f \in \O(\A_R)$ %é holomorfa no anel $\A_R$,
% $A = \{z\in\C: 1 < |z| < R\}$ com $r<1<R$,
tem a série de Laurent $f(z) = \sum c_n z^n$ para $z\in \A_R$.
Define $F(z) = \sum_{n\geq 0} c_n z^n$ e $H(z) = \sum_{n>0} c_{-n} z^{-n}$ e $G(z) = H(z^{-1}) = \sum_{n>0} c_{-n} z^{n}$.
(o) Verifique que $F,G,H$ são holomorfas em $\A_R$.
(a) Define $g(z) := \frac{1}{2\pi i} \int_{|w|=1} \frac{f(w) dw}{w-z}$
para $z\in A - \{|z|=1\}$. Provar que $g(z) = F(z)$ para $|z|<1$ e $g(z)=H(z)$ para $|z|>1$.
(b) Provar que para qualquer $|z|=1$ temos $f(z) = F(z) + H(\bar{z})$.
\end{restatable}

\begin{restatable}{prob}{p11}\label{p11} % [11]
Para $z\in\C-\Z$
(a) a série $\sum_{n\in\Z} \frac{1}{z+n}$ converge absolutamente?
(b) existe $f(z) = \lim_{n\to\infty} S_n$ onde $S_n = \sum_{k=-n}^n \frac{1}{z+n}$?
(c) é localmente uniforme em $\C-\Z$?
(e) a função $f$ tem polos em $\C$?
(f) a função $f$ tem zeros?
(g) $f(z+1) = f(z)$?
(d) qual é o valor $f(1/4)$?
\end{restatable}

\subsection{Bônus de outubro 18}.

\begin{restatable}{prob}{p12}\label{p12} ($i^i=?$)
Sejam $f,g,h \in \O(\C-(-\infty,0])$ tais que para $u\in\R$ e $x=e^u$
temos $f(x) = x^x$, $g(x) = e^{ui}$, $h(x) = e^{\frac{\pi i}{2} x}$.
Calcular $f(i)$, $g(i)$ e $h(i)$.
\end{restatable}

\begin{restatable}{prob}{p13}\label{p13}
Mostre que $f(z) = \frac{z}{1-\exp(-z)}$ é uma função inteira
e calcule $f^{(101)} (0)$ (cento-primeira derivada em zero).
\end{restatable}


\subsection{Os tipos de problemas}

\begin{enumerate}
\item Determinar tipos de pontos singulares e calcular os resíduos.
\item Calcular a integral.
\item Construir biholomorfismo entre duas regiões.
\item Calcular a soma de série.
\item Decompor uma função em série de Laurent nos anéis com dado centro.
\item Descrever a superfície de Riemann de dada função analítica.
\end{enumerate}

\subsection{Exemplos dos problemas finais}
\setcounter{prob}{0}

\emph{Uma estimativa dos níveis de dificuldade é dada nos parênteses.}

\begin{restatable}{prob}{b1}(17)
Sejam $u_0,\dots,u_n\in\C$ distintas,
$P = \prod_{k=0}^n (z-u_k)$ ,
$f(z)$ uma função, $Q \in \C[z]$ um polinómio de grau $n$ tal que
$\forall k: Q(u_k) = f(u_k)$.
Provar que
$f(z) - Q(z) = \frac{1}{2\pi i} P(z) \int_{|w|=R} \frac{f(w) dw}{P(w) \cdot (w-z)}$
para qualquer $R > \max(u_0,\dots,u_n,z)$.
\end{restatable}

\begin{restatable}{prob}{b2}(20)
Provar que todas as raízes complexas de $\tan(z) = z$ são reais.
\end{restatable}

\begin{restatable}{prob}{b3}(7)
Será que existe uma região $\R-\{0\} \subset D\subset \C$ e $f\in\O(D)$ tal que
$f(x) = \log (x^2) \in \R$ para cada $x\in\R-\{0\}$?
\end{restatable}

\begin{restatable}{prob}{b4} (9)(mestrado)
Provar que o cobrimento universal de um toro analítico com um ponto perfurado
é $\D$.
\end{restatable}

\begin{restatable}{prob}{b5}(11)
Provar que 
$f,g\in\O(\C)$ t.q. $\exp(f(z)) + \exp(g(z)) = 1$
são constantes.
\end{restatable}

\begin{restatable}{prob}{b6}(13)
Provar que existe uma função $f$ meromorfa no $\C$ t.q. $f : \Delta \to \D$ é uma bijeção,
onde $\Delta$ é um triângulo com ângulos $\frac{\pi}{6},\frac{\pi}{3},\frac{\pi}{2}$.
\end{restatable}

\begin{restatable}{prob}{b7}(5)
Caraterizar os pontos singulares de $f(z) = \exp(\tan(1/z))$.
\end{restatable}

\begin{restatable}{prob}{b8}(6)
Achar o número de raízes do polinómio $z^3-z^2+3z+5$ com $\Re(z)>0$.
\end{restatable}

\begin{restatable}{prob}{b9}(3)
Provar que $f\in\O(\C)$ t.q. $|f(z)|\leq C |z|^k$ é um polinómio.
\end{restatable}

\begin{restatable}{prob}{b10}(4)
Achar todas as séries de Laurent nos anéis com centro em zero da função $f(z) = \frac{7z-2}{(z+1)(z-2)}$.
\end{restatable}

\begin{restatable}{prob}{b11}(5)
Provar que $u(x,y) = \Re f(x+iy)$ para $f\in\O(\D)$ $\iff$ 
$\frac{\partial^2 u}{\partial x^2} + \frac{\partial^2 u}{\partial y^2} = 0$.
\end{restatable}


\subsection{Função gama}

\begin{defin}[Função gama de Euler e constante de Euler]
\begin{enumerate}
\item A \emph{função gama de Euler} $\Gamma(z)$ para $z>0$ é definida pela
\begin{equation}
\label{eq:gama}
\Gamma(z) = \int_0^\infty t^z e^{-t} \frac{dt}{t}
\end{equation}
\item A \emph{constante de Euler} $C=0.57\dots\in\R$ é definida pela
\begin{equation}
C := \lim_{n\to\infty} \big(1+\frac12+\dots+\frac{1}{n} - \log(n)\big).
\end{equation}
\end{enumerate}
\end{defin}

\begin{restatable}{problema}{gama1}\label{gama1}
Provar que
\begin{enumerate}
\item $\Gamma(z+1) = z \Gamma(z)$;
\item $\Gamma(z+1+n) = (z+n)\cdot(z+n-1)\cdot\dots\cdot z\cdot\Gamma(z)$ para $n\geq 1$ inteiro;
\item $\dlog \Gamma(z+1) - \dlog \Gamma(z) = \dlog z$,
i.e. $\frac{\Gamma'(z+1)}{\Gamma(z+1)} - \frac{\Gamma'(z)}{\Gamma(z)} = \frac{1}{z}$;
\item a função $\Gamma(z)$ tem uma continuação analítica para $\C-\{0,-1,-2,-3,\dots\}$;
\item $\res_{z=-n} \Gamma(z) dz = \frac{(-1)^n}{n!}$ para $n=0,1,2,\dots$.
\end{enumerate}
\end{restatable}

\begin{restatable}{problema}{gama2}[Hankel]\label{gama2}
Provar que
\begin{equation}
\Gamma(z) = \frac{1}{1-\exp(\dpi z)} \int_C \zeta^{z} e^{-\zeta} \frac{d\zeta}{\zeta},
\end{equation}
onde $C$ é o contorno de Hankel:
$C = [i+\infty,i] + S + [-i,-i+\infty]$, onde $S$ é o semicírculo unitário esquarda
em sentido antihorário.
% Verificar os diferentes contornos, etc
\end{restatable}

\begin{restatable}{problema}{gama3}\label{gama3}
Provar que
\begin{equation}
\frac{\Gamma'(z)}{\Gamma(z)} = - C - \frac{1}{z} - \sum_{n=1}^\infty \big(\frac{1}{n+z} - \frac{1}{n}\big).
\end{equation}
\end{restatable}

\begin{restatable}{problema}{gama4}\label{gama4}
\begin{enumerate}
\item Expressar a soma da série
\[ \zeta(3) := 1 + \frac{1}{2^3} + \frac{1}{3^3} + \frac{1}{4^3} + \dots \]
em termos dos valores $\Gamma'(1)$, $\Gamma''(1)$ e $\Gamma'''(1)$.
\item Expressar os valores $C,\zeta(2),\zeta(3),\zeta(4),\zeta(5),\dots$
em termos dos coeficientes de Taylor de $\Gamma(z)$ em $z=1$ e vice versa.
\end{enumerate}
\end{restatable}

% 


