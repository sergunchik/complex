\subsection*{Notações.}
As regiões comuns:
$\Ch = \C \bigcup \infty$ --- a \emph{esfera de Riemann},
$H = \{z\in\C: \Im(z)>0\}$ --- o \emph{plano superior},
$D_R = \{z\in\C: |z|<R\}$ --- o \emph{disco} de raio $R>0$,
$\D$ --- o \emph{disco unitário},
$\D^* = \D - \{0\}$ --- o \emph{disco perfurado},
$\A_R := \{z\in\C: 1/R < |z| < R \}$ o \emph{anel} para $R\in(1,+\infty)$,
$D\subset \Ch$ --- uma região;
$\O(D)$ --- o conjunto das funções holomorfas na $D$;
$\phi : \Ch\to\Ch$ --- as \emph{transformações de Möbius} 
$\phi(z) = \frac{az+b}{cz+d}$
com $a,b,c,d\in\C$ e $ad\neq bc$.

\subsection{Outubro 18}

\begin{restatable}{prob}{p1}\label{p1} % [1]
Para $x\in(-1,1)$
construir um biholomorfismo %elementar entre
$\psi: \H \to \D - [x,1)$.
% o plano superior
% e o complemento para o intervalo $[x,1) \subset \R$
% no disco unitário.
\end{restatable}

\begin{restatable}{prob}{p2}\label{p2} % [2]
Para círculo generalizado (círculo ou reta) $S \subset \Ch$ define
uma \emph{reflexão}
$R_S : \Ch \to \Ch$ como a reflexão em reta ou a inversão em círculo.
% Seja $\phi: \Ch \to \Ch$ uma transformação de Möbius, i.e. $\phi(z) = \frac{az+b}{cz+d}$
%com $ad\neq bc$.
Provar que as transformações de Möbius $\phi : \Ch\to\Ch$
preservam as reflexões, i.e. para qualquer
círculo $S$ e a imágem dela $\phi(S)$ temos $\phi(R_S z) = R_{\phi(S)} \phi(z)$.
\end{restatable}

\begin{restatable}{prob}{p3}\label{p3} % [3]
Sejam $C_1,C_2 \subset \Ch$ um par de círculos disjuntos, i.e. $C_1 \bigcap C_2 = \emptyset$,
e $D \subset \Ch$ uma região com a fronteira $\partial D = C_2 - C_1$.
(a) Provar que existe uma transformação de Möbius $\phi$ tal que
$\phi(D) = \A_R$ para algum $R>1$. %  = \{ z\in\C: 1/R < |z| < R \}$.
(b) Qual é o lugar geométrico de círculos $S$, que tocam ambos $C_1$ e $C_2$?
\end{restatable}

\begin{restatable}{prob}{p4}\label{p4} % [4]
Define $Sf = \frac{f'''}{f'} - \frac{3}{2} \big(\frac{f''}{f'}\big)^2$.
Provar para qualquer funções $f,g$ com singularidades isoladas e qualquer
transformação de Möbius $\phi$ que
(a) $S \phi = 0$,
(b) $S (\phi \circ f) = S f$,
(c) $S(f \circ \phi) = \big((Sf)\circ\phi\big) \cdot (\phi')^2$,
(d) $S(f\circ g) = \big((Sf)\circ g\big) \cdot (g')^2 + Sg$.
\end{restatable}

\begin{restatable}{prob}{p5}\label{p5} % [5]
Achar $\lim_{\rho\to+0} \int_\alpha^\beta f(\rho\cdot e^{i\phi}) d\phi$
para $f : D_\epsilon\to\C$ contínua.
\end{restatable}

\begin{restatable}{prob}{p6}\label{p6} % [6]
Seja $\omega_t = \frac{e^{itz} dz}{(z+z^{-1})^3}$.
(a) Classificar singularidades de $\omega_t$ na $\Ch$
para cada $t\in\C$.
(b) Calcular $I(t) = \int_\R \omega_t$.
\end{restatable}

\begin{restatable}{prob}{p7}\label{p7} % [7]
Para $f\in\O(\D^*)$ % uma função holomorfa no disco perfurado $D^*$. % = P(0,1) = \{z\in\C: 0<|z|<1\}$.
suponha que $\frac{f'(z)}{f(z)}$
tem um polo simples em $0$. Provar que
(a) a função $f$ é meromorfa em $0$,
(b) e $f$ tem zero ou polo em $0$.
\end{restatable}

\begin{restatable}{prob}{p8}\label{p8} % [8]
Seja $f(z) = \frac{P(z)}{Q(z)}$ para um par de polinómios $P,Q\in\C[z]-\{0\}$.
Suponha que para qualquer $|z|=1$ temos $|f(z)|=1$. Classificar $f$,
i.e. dar uma condição suficiente e necessária sobre os polos e nulos de $f$.
\end{restatable}

\begin{restatable}{prob}{pnove}\label{p9} % [9]
%Quais podem ser as funções inteiras 
Classificar $f,g\in\O(\C)$ tais que
$f^2+g^2=1$ e $f(z+w) = f(z) g(w) + f(w) g(z)$.
\end{restatable}

\begin{restatable}{prob}{p10}\label{p10} % [10]
Suponha que $f \in \O(\A_R)$ %é holomorfa no anel $\A_R$,
% $A = \{z\in\C: 1 < |z| < R\}$ com $r<1<R$,
tem a série de Laurent $f(z) = \sum c_n z^n$ para $z\in \A_R$.
Define $F(z) = \sum_{n\geq 0} c_n z^n$ e $H(z) = \sum_{n>0} c_{-n} z^{-n}$ e $G(z) = H(z^{-1}) = \sum_{n>0} c_{-n} z^{n}$.
(o) Verifique que $F,G,H$ são holomorfas em $\A_R$.
(a) Define $g(z) := \frac{1}{2\pi i} \int_{|w|=1} \frac{f(w) dw}{w-z}$
para $z\in A - \{|z|=1\}$. Provar que $g(z) = F(z)$ para $|z|<1$ e $g(z)=H(z)$ para $|z|>1$.
(b) Provar que para qualquer $|z|=1$ temos $f(z) = F(z) + H(\bar{z})$.
\end{restatable}

\begin{restatable}{prob}{p11}\label{p11} % [11]
Para $z\in\C-\Z$
(a) a série $\sum_{n\in\Z} \frac{1}{z+n}$ converge absolutamente?
(b) existe $f(z) = \lim_{n\to\infty} S_n$ onde $S_n = \sum_{k=-n}^n \frac{1}{z+n}$?
(c) é localmente uniforme em $\C-\Z$?
(e) a função $f$ tem polos em $\C$?
(f) a função $f$ tem zeros?
(g) $f(z+1) = f(z)$?
(d) qual é o valor $f(1/4)$?
\end{restatable}

\subsection{Bônus de outubro 18}.

\begin{restatable}{prob}{p12}\label{p12} ($i^i=?$)
Sejam $f,g,h \in \O(\C-(-\infty,0])$ tais que para $u\in\R$ e $x=e^u$
temos $f(x) = x^x$, $g(x) = e^{ui}$, $h(x) = e^{\frac{\pi i}{2} x}$.
Calcular $f(i)$, $g(i)$ e $h(i)$.
\end{restatable}

\begin{restatable}{prob}{p13}\label{p13}
Mostre que $f(z) = \frac{z}{1-\exp(-z)}$ é uma função holomorfa em $0$
e calcule $f^{(101)} (0)$ (cento-primeira derivada em zero).
\end{restatable}


\subsection{Os tipos de problemas}

\begin{enumerate}
\item Determinar tipos de pontos singulares e calcular os resíduos.
\item Calcular a integral.
\item Construir biholomorfismo entre duas regiões.
\item Calcular a soma de série.
\item Decompor uma função em série de Laurent nos anéis com dado centro.
\item Descrever a superfície de Riemann de dada função analítica.
\end{enumerate}

\subsection{Exemplos dos problemas finais}
\setcounter{prob}{0}

\emph{Uma estimativa dos níveis de dificuldade é dada nos parênteses.}

\begin{restatable}{prob}{b1}(17)
Sejam $u_0,\dots,u_n\in\C$ distintas,
$P = \prod_{k=0}^n (z-u_k)$ ,
$f(z)$ uma função, $Q \in \C[z]$ um polinómio de grau $n$ tal que
$\forall k: Q(u_k) = f(u_k)$.
Provar que
$f(z) - Q(z) = \frac{1}{2\pi i} P(z) \int_{|w|=R} \frac{f(w) dw}{P(w) \cdot (w-z)}$
para qualquer $R > \max(u_0,\dots,u_n,z)$.
\end{restatable}

\begin{restatable}{prob}{b2}(20)
Provar que todas as raízes complexas de $\tan(z) = z$ são reais.
\end{restatable}

\begin{restatable}{prob}{b3}(7)
Será que existe uma região $\R-\{0\} \subset D\subset \C$ e $f\in\O(D)$ tal que
$f(x) = \log (x^2) \in \R$ para cada $x\in\R-\{0\}$?
\end{restatable}

\begin{restatable}{prob}{b4} (9)(mestrado)
Provar que o cobrimento universal de um toro analítico com um ponto perfurado
é $\D$.
\end{restatable}

\begin{restatable}{prob}{b5}(11)
Provar que 
$f,g\in\O(\C)$ t.q. $\exp(f(z)) + \exp(g(z)) = 1$
são constantes.
\end{restatable}

\begin{restatable}{prob}{b6}(13)
Provar que existe uma função $f$ meromorfa no $\C$ t.q. $f : \Delta \to \D$ é uma bijeção,
onde $\Delta$ é um triângulo com ângulos $\frac{\pi}{6},\frac{\pi}{3},\frac{\pi}{2}$.
\end{restatable}

\begin{restatable}{prob}{b7}(5)
Caraterizar os pontos singulares de $f(z) = \exp(\tan(1/z))$.
\end{restatable}

\begin{restatable}{prob}{b8}(6)
Achar o número de raízes do polinómio $z^3-z^2+3z+5$ com $\Re(z)>0$.
\end{restatable}

\begin{restatable}{prob}{b9}(3)
Provar que $f\in\O(\C)$ t.q. $|f(z)|\leq C |z|^k$ é um polinómio.
\end{restatable}

\begin{restatable}{prob}{b10}(4)
Achar todas as séries de Laurent nos anéis com centro em zero da função $f(z) = \frac{7z-2}{(z+1)(z-2)}$.
\end{restatable}

\begin{restatable}{prob}{b11}(5)
Provar que $u(x,y) = \Re f(x+iy)$ para $f\in\O(\D)$ $\iff$ 
$\frac{\partial^2 u}{\partial x^2} + \frac{\partial^2 u}{\partial y^2} = 0$.
\end{restatable}

\subsection{Miscelânea}

\begin{prob}
Suponha que $f(z) = f(x+yi) = f(x,y)$ tem a derivada real em $z_0$.
Provar que o número complexo $c\in\C$ pode ser o limite
$\lim_{z_n\to z_0} \frac{f(z_n) - f(z_0)}{z_n-z_0} = c$
se e somente se $c$ é no círculo com centro
$\frac{\partial f}{\partial z} = \frac{1}{2}\big(\frac{\partial f}{\partial x} - i\frac{\partial f}{\partial y}$
e raio
$|\frac{\partial f}{\partial \ol{z}}|$
onde
$\frac{\partial f}{\partial \ol{z}} = \frac{1}{2}\big(\frac{\partial f}{\partial x} + i\frac{\partial f}{\partial y}$.
\end{prob}



\subsection{Transformadas de Möbius}

\begin{defin}
Uma \emph{transformação de Möbius} é uma bijeção da esfera de Riemann $\Ch$ dada por
$z\mapsto \frac{az+b}{cz+d}$ com $a,b,c,d\in\C$ e $ad - bc \neq 0$ .
\end{defin}

\begin{problema}
Provar que
\begin{enumerate}
\item O grupo das transformações de Möbius é gerado por $z\mapsto \lambda z$,
$z\mapsto z+1$ e $z\mapsto 1/z$.
\item As transformações de Möbius mandem os círculos generalizados (círculos ou retas)
nos círculos generalizados.
\item As transformações de Möbius respeitam as reflexões:
as imagens de pontos simétricos com respeito de um círculo são simétricos com respeito à imagem dele.
\item As transformações de Möbius são conformes, i.e. preservam os ângulos.
\end{enumerate}
\end{problema}

\begin{problema}
Provar que um par de pontos $z_1,z_2\in \Ch$ são simétricos com respeito de um círculo $C$
se e somente se qualquer círculo que passa de $z_1$ e $z_2$ intersecta o círculo $C$
nos ângulos retos.
\end{problema}

\begin{problema}
Provar que por qualquer par de círculos em $\Ch$ existe uma transformada de Möbius que os manda no
um dos seguintes
\begin{enumerate}
\item par de círculos concêntricos,
\item ou par de retas paralelas,
\item ou par de retas passando de zero.
\end{enumerate}
\end{problema}

\begin{problema}
Sejam $C$ um círculo com centro $O$ e $C' = \phi(C)$ a imagem dela,
o círculo com centro $O'$. Dar um exemplo em qual $O' \neq \phi(O)$.
\end{problema}

\begin{problema}[Círculos de Apolônio]
Sejam $z_1,z_2\in\C$. Provar que por cada $K>0$ o lugar de pontos
$S_K = \{z \in \C: \frac{|z-z_1|}{|z-z_2|} = K \}$ é um círculo, chamado um círculo de Apolônio.
Provar que qualquer círculo $S_K$ interseta qualquer círculo que passa de $z_1$ e $z_2$ em ângulos retos.
Desenha essas duas famílias de círculos.
\end{problema}

\begin{problema}
Sejam $C$ e $D$ um par de círculos disjuntos. Suponha que há $N$ círculos
$S_1,\dots,S_N$ tais que cada $S_i$ e tangente a $C$ e a $D$,
e também $S_i$ é tangente a $S_{i+1}$ e $S_N$ é tangente a $S_1$ (pode desenhar-las em sentido anti-horário).
Provar que para qualquer outros $N$ círculos $S_1',\dots,S_N'$ tais que
cada $S_i'$ é tangente aos $C$ e $D$ e também $S_1'$ é tangente ao $S_2'$,\dots,$S_{N-1}'$ é tangente ao $S_N'$
temos necessariamente que $S_N'$ é tangente ao $S_1'$.
\end{problema}

\begin{problema}
Sejam 
$z_0,z_1,z_\infty \in\Ch$ tais que $z_0\neq z_1$, $z_0\neq z_\infty$, $z_1\neq z_\infty$
e 
$w_0,w_1,w_\infty \in\Ch$ tais que $w_0\neq w_1$, $w_0\neq w_\infty$, $w_1\neq w_\infty$.
Provar que existe a única transformada de Möbius $\phi : \Ch \to \Ch$ tal que
$\phi(z_0) = w_0$, $\phi(z_1) = w_1$ e $\phi(z_\infty) = w_\infty$.
\end{problema}

\begin{problema}
Desenhe as imagens das retas verticais $\Re z = x$ e horizontais $\Im z = y$
com respeito de uma transformada de Cayley $w = \frac{z-i}{z+i}$.
\end{problema}

\begin{problema}
Desenhe as imagens dos círculos que passam de pontos $z=i$ e $z=-i$ com respeito da transformada
de Caylet $w = \frac{z-i}{z+i}$.
\end{problema}

\begin{problema}
Ache $\phi(D)$ para $D$ e $w=\phi(z)$ dado por
\begin{enumerate}
\item $D = \{z\in\C: \Im(z) = 1\}$, $\phi(z) = \frac{z-1}{z+1}$.
\item $D = \{z\in\C: |z-1/2| = 1/4\}$, $w = -1/z$.
\item $D = \{z\in\C: |z|\geq 1 \}$, $w = 2 \frac{2z-1}{z-2}$.
\end{enumerate}
\end{problema}

\subsection{Funções elementares}
\label{ss:elementar}

Lembra as definições das funções trigonométricas seno $\sin$, cosseno $\cos$,
tangente $\tan$ e cotangente $\cot$,
e as versões hiperbólicas $\sinh, \cosh, \tanh, \coth$ delas:
\begin{align*}
\cos(z) &:= \frac{\exp(iz) + \exp(-iz)}{2}, &
\cosh(z) &:= \frac{\exp(z) + \exp(-z)}{2}, \\
\sin(z) &:= \frac{\exp(iz) - \exp(-iz)}{2i}, &
\sinh(z) &:= \frac{\exp(z) - \exp(-z)}{2}, \\
\tan(z) &:= \frac{\sin(z)}{\cos(z)}, &
\tanh(z) &:= \frac{\sinh(z)}{\cosh(z)}, \\
\cot(z) &:= \frac{\cos(z)}{\sin(z)}, &
\coth(z) &:= \frac{\cosh(z)}{\sinh(z)}.
\end{align*}

\begin{problema}
Provar que
\begin{enumerate}
\item $\exp(z) = 1 \iff z\in 2\pi i\Z$,
\item $\exp(z) = \exp(w) \iff z-w \in 2\pi i\Z$,
\item $\cos(z) = \cos(w)$ se e só se $z-w \in 2\pi\Z$ ou $z+w \in 2\pi\Z$,
\item $\sin(z) = \sin(w)$ se e só se $z-w \in 2\pi\Z$ ou $z+w \in \pi(1+2\Z)$,
i.e. $w=z+\pi (2k)$ ou $w = -z+\pi(2k+1)$ para $k\in\Z$.
\item $\tan(z) = \tan(w)$ (ou $\cot(z) = \cot(w)$)
se e só se $z-w\in\pi\Z$,
\item análogos para equação $f(z) = f(w)$ onde $f = \sinh, \cosh, \tanh, \coth$.
\end{enumerate}
\end{problema}

\begin{problema}
Provar que existem $r,R>0$ e uma série $g(w) = w + \sum_{n>1} c_n w^n$ convergente em $B(0,R)$
tal que para todos $w\in B(0,R)$ e $z\in B(0,r)$ a função $g(w)$ é resersa da função $f(z)$
para $f\in \{\sin,\tan\}$, i.e.
\begin{enumerate}
\item $\sin(g(w)) = w$ e $g(\sin(z)) = z$,
\item $\tan(g(w)) = w$ e $g(\tan(z)) = z$.
\end{enumerate}
Quais são os valores supremos de $r$ e $R$ tal que essa função reversa $g$ existe?\footnote{Dica:
se uma função $f(z)$ tem um periodo $p\in\C$ (i.e. $f(z+p)=f(z)$),
então a restrição dela no $B(0,r)$
não pode ser injetora para $r>|p|/2$.}
\end{problema}

\begin{defin}
No último problema a função reversa de $w=\sin(z)$ se chama \emph{arcseno} $z = \arcsin(w)$,
e a função reversa de $w=\tan(z)$ se chama \emph{arctangente} $z = \arctan(w)$.
A priori as funções $\arcsin(w)$ e $\arctan(w)$ só definidos em vizinhanças de $w=0$.
\end{defin}

\begin{problema}
Quais são as séries de Taylor--Maclaurin (ou Laurent) em $0$ das funções
\begin{enumerate}
\item $\exp(z)$, $\sin(z)$, $\cos(z)$, $\sinh(z)$, $\cosh(z)$,
\item $\tan(z)$ e $\tanh(z)$,
\item $\arcsin(w)$,
\item $\arctan(w)$,
\item $\cot(z)$ e $\coth(z)$ (as séries de Laurent)?
\end{enumerate}
\end{problema}


\subsection{Resíduos}

Veja \Cref{d:residuo} para uma definição e discussão de resíduos.

\begin{problema}
Provar que
\begin{enumerate}
\item $\res_{z=z_0} f(z) = c_{-1}$ onde $f(z) = \sum_{n\in\Z} c_n (z-z_0)^n$
em vizinhança perfurada de $z_0$;
\item $\res_{z=\infty} f(z) = - c_{-1}$ onde $f(z) = \sum_{n\in\Z} c_n (z-a)^n$
por algum/cada $a\in\C$ em vizinhança de $\infty$;
\item $\res_{z=\infty} f(z) = \res_{w=0} \frac{-f(w)}{w^2}$;
\item $\int_{\partial D} f(z) dz = \sum_{w\in D} \res_{z=w} f(z)$;
\item $\res_{z=z_0} f(z) = \lim_{z\to z_0} (z-z_0) f(z)$ se $f$ tem um polo simples em $z_0\in\C$;
\item Se $f(z) = \frac{\phi(z)}{\psi(z)}$ com $\psi(z_0)=0$ e $\psi'(z_0)\neq 0$
então $\res_{z=z_0} f(z) = \frac{\phi(z_0)}{\psi'(z_0)}$;
\item se $f(z)$ tem um polo de ordem $k$ em $z_0\in\C$ então
$\res_{z=z_0} f(z) = \frac{1}{(k-1)!} g^{(k-1)}(z_0)$ onde $g(z)$ é uma função holomorfa em $z_0$
tal que $g(z) = (z-z_0)^k f(z)$.
\end{enumerate}
\end{problema}

\begin{problema}
Computar os resíduos
\begin{enumerate}
\item $\res_{z=\pm \pi/2} \frac{1}{\cos(z)}$;
\item $\res_{z=w} \frac{1}{\sin(z)^3}$ para $w=0,\pi$;
\item $\res_{z=ib} \frac{\exp(iaz) \cdot (z^2-b^2)}{z(z^2+b^2)}$;
\item $\res_{z=w} \frac12 (z+z^{-1})$ para $w=0,\infty$;
\item $\res_{z=0} (z-1)^2 \exp(-1/z^2)$;
\item $\res_{z=\infty} z^2 \sin(\exp(1/z))$;
\item $\res_{z=1} \frac{1}{1+\sqrt{z}}$.
\end{enumerate}
\end{problema}

\begin{problema}
Calcular as integrais $\int_{\partial D} f(z) dz$ para
\begin{enumerate}
\item $\int_{|z-1-i|=2} \frac{dz}{(z-1)^2 (z^2+1)}$;
\item $\int_{|z|=2} z \sin(\frac{z+1}{z-1}) dz$;
\item $\int_{|z|=3} \sin(\frac{z}{z+1}) dz$.
\end{enumerate}
\end{problema}


\begin{problema}
Seja $R \in \C(x,y)$ uma função racional tal que o valor $R(\cos(t),\sin(t))\in\C$ é bem definido
para qualquer $t\in\R$. Provar que
\[ \int_0^{2\pi} R(\cos t,\sin t) dt 
 = 2\pi \sum_{|w|<1} \res_{z=w} z^{-1} R(\frac{z+z^{-1}}{2},\frac{z-z^{-1}}{2i}) \]
\end{problema}

\begin{problema}
Seja $R \in \C(x)$ uma função racional tal que $R(x)\in\C$ é bem definido para qualquer $x\in\R$
e $\lim_{z\to\infty} zR(z) = 0$. Provar que
\begin{enumerate}
\item a integral $\int_{-M}^N R(x) dx$
converge para $M,N\to+\infty$,
\item  e \[ \int_{-\infty}^\infty R(x) dx = 2\pi i \sum_{\Im(w)>0} \res_{z=w} R(z). \]
\end{enumerate}
\end{problema}

\begin{problema}
Seja $f(z)$ uma função holomorfa em $\Im(z)\geq 0$ exceto um conjunto finito de pontos no interior,
e tal que $\lim_{\Im(z)>0, |z|\to\infty} f(z) = 0$.
\begin{enumerate}
\item Para $\lambda>0$ provar que
\[ \int_{-\infty}^\infty e^{i\lambda z} f(z) dz = 2\pi i \sum_{\Im(w)>0} \res_{z=w} e^{i\lambda z} f(z); \]
\item Como calcular $\int_{-\infty}^\infty e^{i\lambda z} f(z) dz$ para $\lambda<0$?
\end{enumerate}
\end{problema}

\begin{problema}
Seja $R \in \C(x)$ uma função racional tal que $R(x)\in\C$ é bem definido para qualquer $x\in\R$
e $\lim_{z\to\infty} R(z) = 0$. Provar que para $0<\alpha<1$
\[ \int_{-\infty}^\infty \frac{R(z)}{z^\alpha} dz 
 = \frac{2\pi i}{1-\exp(-2\pi i\alpha)} \sum_{w\in\C} \res{z=w} \frac{R(z)}{z^\alpha}. \]
\end{problema}

\begin{problema}
Calcular as integrais
\begin{enumerate}
\item $\int_0^{2\pi i} \frac{\sin^2(x)}{5+3\cos(x)} dx$;
\item $\int_{-\infty}^\infty \frac{x^2}{x^4 + 6 x^2 + 25} dx$;
\item $\int_{0}^{\infty} \frac{x^2 - b^2 \sin(ax)}{x^2 + b^2} dx$;
\item $\int_{0}^{\infty} \frac{x^\alpha}{1+x}  \frac{dx}{x}$ para $0<\alpha<1$;
\item $\int_{0}^{\infty} \frac{\log(x)}{(x^2+1)^2} dx$;
\item $\int_{0}^{\infty} \frac{\log(x)}{(x+a)^2+b^2} dx$;
\item $\int_{0}^{\infty} \frac{\cos(x)}{x^2 + a^2}  dx$ (Laplace);
\item $\int_{0}^{\infty} \frac{\sin(x)}{x}  dx$ (Euler);
\item $\int_{-\infty}^{\infty} \frac{\exp(ax)}{1+\exp(x)} dx$ (Euler);
\item $\int_{0}^{\infty} \exp(-a x^2) \cos(bx)  dx$ (Poisson);
\item $\int_{0}^{\infty} \exp(-\pi x) \frac{\sin(ax)}{\sinh(\pi x)}  dx$ (Legendre).
\end{enumerate}
\end{problema}

\begin{problema}
Seja $f \in \C(z)$ uma função racional com polos $a_1,\dots,a_k \in \C -\Z$ tal
que $f(z) = O(z^{-2})$ para $z\to\infty$. Provar que
\[ \sum_{n\in\Z} f(n) = -\pi \sum \res_{z=a_j} f(z) \cot(\pi z) . \]
\end{problema}

\begin{problema}
Seja uma função meromorfa no $\C$ com finito número de polos $a_1,\dots,a_m$
que não são inteiros, e tal que em $\{z\in\C : |z-a_k|\geq r\}$ temos
$|f(z)| < \exp(a|\Im(z)|) g(|z|)$ onde $0<a<\pi$ e $\lim_{t\to\infty} g(t) = 0$.
Provar que 
\[ \sum_{n\in\Z} (-1)^n f(n) =  -\pi \sum \res_{z=a_k} \frac{f(z)}{\sin(\pi z)} . \]
\end{problema}


\begin{problema}
Calcular as somas
\begin{enumerate}
\item $\sum_{n\in\Z} \frac{1}{(n-a)^2}$,
\item $\sum_{n=1}^\infty \frac{1}{n^2 + a^2}$,
\item $\sum_{n=1}^\infty \frac{1}{n^2}$,
\item $\sum_{n=1}^\infty \frac{(-1)^n}{(n^2+a^2)^2}$,
\item $\sum_{n=1}^\infty \frac{(-1)^n}{n^3}$.
\end{enumerate}
\end{problema}

\subsection{Função gama}

\begin{defin}[Função gama de Euler e constante de Euler]
\begin{enumerate}
\item A \emph{função gama de Euler} $\Gamma(z)$ para $z>0$ é definida pela
\begin{equation}
\label{eq:gama}
\Gamma(z) = \int_0^\infty t^z e^{-t} \frac{dt}{t}
\end{equation}
\item A \emph{constante de Euler} $C=0.57\dots\in\R$ é definida pela
\begin{equation}
C := \lim_{n\to\infty} \big(1+\frac12+\dots+\frac{1}{n} - \log(n)\big).
\end{equation}
\end{enumerate}
\end{defin}

\begin{restatable}{problema}{gama1}\label{gama1}
Provar que
\begin{enumerate}
\item $\Gamma(z+1) = z \Gamma(z)$;
\item $\Gamma(z+1+n) = (z+n)\cdot(z+n-1)\cdot\dots\cdot z\cdot\Gamma(z)$ para $n\geq 1$ inteiro;
\item $\dlog \Gamma(z+1) - \dlog \Gamma(z) = \dlog z$,
i.e. $\frac{\Gamma'(z+1)}{\Gamma(z+1)} - \frac{\Gamma'(z)}{\Gamma(z)} = \frac{1}{z}$;
\item a função $\Gamma(z)$ tem uma continuação analítica para $\C-\{0,-1,-2,-3,\dots\}$;
\item $\res_{z=-n} \Gamma(z) dz = \frac{(-1)^n}{n!}$ para $n=0,1,2,\dots$.
\end{enumerate}
\end{restatable}

\begin{restatable}{problema}{gama2}[Hankel]\label{gama2}
Provar que
\begin{equation}
\Gamma(z) = \frac{1}{1-\exp(\dpi z)} \int_C \zeta^{z} e^{-\zeta} \frac{d\zeta}{\zeta},
\end{equation}
onde $C$ é o contorno de Hankel:
$C = [i+\infty,i] + S + [-i,-i+\infty]$, onde $S$ é o semicírculo unitário esquarda
em sentido antihorário.
% Verificar os diferentes contornos, etc
\end{restatable}

\begin{restatable}{problema}{gama3}\label{gama3}
Provar que
\begin{equation}
\frac{\Gamma'(z)}{\Gamma(z)} = - C - \frac{1}{z} - \sum_{n=1}^\infty \big(\frac{1}{n+z} - \frac{1}{n}\big).
\end{equation}
\end{restatable}

\begin{restatable}{problema}{gama4}\label{gama4}
\begin{enumerate}
\item Expressar a soma da série
\[ \zeta(3) := 1 + \frac{1}{2^3} + \frac{1}{3^3} + \frac{1}{4^3} + \dots \]
em termos dos valores $\Gamma'(1)$, $\Gamma''(1)$ e $\Gamma'''(1)$.
\item Expressar os valores $C,\zeta(2),\zeta(3),\zeta(4),\zeta(5),\dots$
em termos dos coeficientes de Taylor de $\Gamma(z)$ em $z=1$ e vice versa.
\end{enumerate}
\end{restatable}

% 


