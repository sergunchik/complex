
Lembranças sobre os números complexos antes de começar a análise.

Números complexos $\C$ são munidos com as operações binárias
de soma e produto $+,\cdot : \C \times \C \to \C$,
e com as funções de 
conjugação $\bar{\cdot} : \C \to \C$,
parte real e parte imaginária $\Re,\Im : \C \to \R$,
norma e valor absoluto $N,|\cdot| : \C \to \R_{\geq 0}$,
argumento $\arg : \C \to \R/2\pi\Z$,
valor principal de argumento $Arg : \C \to [0,2\pi)$.

\begin{gather}
z = x + y i \\
z = r (\cos \phi + i \sin \phi) = r e^{i\phi}\\
\bar{z} = x - y i = r e^{-i\phi} \\
x = r \cos \phi \\
y = r \sin \phi \\
N(z) = z \bar{z} = r^2 = x^2+y^2 \\
|z| = \sqrt{N(z)} = r \\
\arg z = \phi \mod 2\pi
\end{gather}

A norma é multiplicativa:
$$ N(zw) = N(z) N(w) $$

Podemos munir $\C$ com a função de distância $d(z,w) := |z-w|$.
Ele é uma métrica, i.e. $d(z,w) \geq 0$, $d(z,w) = 0$ se e só se $z=w$,
$d(z,w) = d(w,z)$ e $d(z,w) \leq d(z,s) + d(s,w)$ (desigualdade de triângulo).

Pelo teorema de Pitágoras essa métrica coincide com a métrica euclidiana.

Similarmente podemos munir $\C$ com uma orientação, declarando que o vetor $i v$ é no lado esquerda de vetor $v$.

Multiplicatividade de norma implica que $d(az+b,aw+b) = |a| d(z,w)$, i.e. se $|a|=1$ a transformação $z\mapsto az+b$
é uma isometria (preserva as distâncias) euclidiana.

Então geometricamente (Caspar Wessel 1799, e depois Gauss e Argand) 
podemos identificar $\C$ com $\R^2$ e com um plano euclidiano.
As transformações $z \to z + b$ são translações,
as transformações $z \to (\cos \phi + i \sin \phi) \cdot z$ são rotações com centro em $0$,
as transformações $z \to r \cdot z$ ($r\in\R_{>0}$) são dilatações. 
Todas elas preservam a orientação e também são conformes, i.e. preservam os ângulos.
Também todas $z \to (\cos \phi + i \sin \phi) \cdot z + b$ preservam as distâncias euclidianas.
As transformações $z \to (\cos \phi + i \sin \phi) \cdot \bar{z} + b$ coincidem com as reflexões
em diferentes retas, elas preservam as distâncias, mas trocam a orientação.

O subconjunto $U \subset \C$ é chamado \emph{aberto}\footnote{com respeito de topologia euclidiana}
se ele é uma união de bolas abertas $B(z,r) := \{w\in \C t.q. |w-z|<r\}$ ($z\in\C,r\in\R$), 
i.e. $\forall z\in U \exists r>0 \textit{tal que} \forall w\in\C |w-z|<r \implies w\in U$,
em palavras -- qualquer ponto de $U$ também tem uma vizinhança dele dentro de $U$.

Na análise complexa as palavras \emph{região} ou \emph{domínio} significam um subconjunto aberto de $\C$.
Também em seguida geralmente temos em mente subconjuntos \emph{conexos}, i.e. tais
que não são uniões disjuntas de dois subconjuntos abertos não vazios.
O propriedade que será mais útil para nós é \emph{conexidade em caminhos}:
para qualquer par de pontos $P,Q\in U$ existe um caminho em $U$ entre eles,
i.e. existe uma função contínua $\gamma: [0,1] \to U$ tal que $\gamma(0) = P$
e $\gamma(1)=Q$. Se quiser pode se imaginar que essa função é linear em pedaços,
ou ainda feita só de intervalos horizontais e verticais.

Muitas definições e teoremas básicas sobre as funções de variável real
podem ser promovidas para as funções de variável complexa usando
a interpretação de $|z|$ acima.

\section{Diferenciação complexa}

\subsection{Funções de variável complexa}

Sejam $D\subset \C$ uma região e $f : D \to \C$ uma função,
$D\supset z = x + yi$, $f(z) = w = u + v i$ para $x,y,u,v\in \R$, $z\in D\subset \C$, $w\in \C$.
Podemos também interpretar $f$ como uma função de 1 variável complexa com um valor complexo,
ou como uma função $u(x,y) + v(x,y) i : \R^2\supset D \to \C$ de $2$ variáveis reais com um valor complexo,
ou como um par de funções de $2$ variáveis reais
$u,v : D \to \R$, $u(x,y), v(x,y) \in \R$,
i.e. $(x,y) \mapsto (u(x,y),v(x,y)) : \R^2 \supset D \to \R^2$.

Para $z$ complexo o limite $\lim_{z\to 0}$ significa $\lim_{|z|\to 0}$.
Similarmente $\lim_{z\to z'}$ é mesmo que $\lim_{|z-z'|\to 0}$.

\subsection{Derivada complexa}
Em seguida definição $\cdot$ é multiplicação em $\C$,
geralmente omitimos este símbolo, mas quero enfatizar que essa definição depende
de multiplicação (resp. divisão) de números complexos.
\begin{defin}
Suponha que existe $a\in \C$ tal que
$$ f(z) = f(z_0) + a \cdot (z-z_0) + g(z) \cdot (z-z_0) $$
tal que $\lim_{z\to z_0} g(z) = 0$,
i.e. $\lim_{z\to z_0} \frac{f(z)-f(z_0)}{z-z_0} = a$.
Neste caso digamos que $f$ tem a derivada complexa em $z_0$ igual ao $a\in \C$
e denotamos $f'(z_0) := a$, alternativamente usamos notação $\frac{df}{dz}(z_0) = f'(z_0)$.
Em linguagem $\epsilon-\delta$:
$\forall\epsilon>0\exists\delta>0$ t.q. $|z-z_0|<\delta$ implica
$|f(z)-f(z_0)-a(z-z_0)| < \epsilon |z-z_0|$.
\end{defin}

\begin{lema}
Se $f(z) = u(x,y) + iv(x,y)$ tem a derivada complexa em ponto $z_0$ neste ponto temos as equações de Cauchy--Riemann
para derivadas parciais: 
\begin{equation}
u_x = v_y, \,\,\, u_y = - v_x
\end{equation}
onde $u_x = \frac{\partial u(x,y)}{\partial x} = \lim_{\epsilon\to 0} \frac{u(x+\epsilon,y)-u(x,y)}{\epsilon}$,etc
são as derivadas parciais.
\end{lema}

\emph{Diferenciais complexas} (formas diferenciais complexas).
Em notação-definição $\frac{df}{dz}(z_0) = f'(z_0)$ podemos formalmente multiplicar pelo denominador para obter
uma equação $(df)(z_0) = f'(z_0) (dz)(z_0)$.
Por enquanto podemos formalmente definir um espaço vetorial complexo $1$-dimensional com um vetor-base $(dz)(z_0)$ e para cada função $f(z)$ com derivada complexa em ponto $z_0$ associar outro vetor $(df)(z_0) := f'(z_0) (dz)(z_0)$. Essa notação (de Leibniz)
será útil ao menos para escrever as fórmulas mais convenientes.

Notações $\partial_z$ e $\partial_{\bar{z}}$.
Equações de Cauchy--Riemann são equivalentes ao equação $\partial_{\bar{z}} f(z_0) = 0$,
neste caso $\partial_z f(z_0) = f'(z_0)$.

\begin{lema}
Se $f$ tem derivada complexa $f'(z_0) = r (\cos\phi + i \sin \phi)$ em ponto $z_0$
ela também tem derivada como aplicação
de $D \subset \R^2$ para $R^2$, e ela é igual a
$\begin{smallmatrix} u_x & -v_y & u_y & v_x \end{smallmatrix}
= r \begin{smallmatrix} \cos\phi & -\sin\phi & \sin\phi & \cos\phi \end{smallmatrix} $.
\end{lema}
\begin{proof}
Lembra que geralmente para aplicação $F: \R^n \to \R^m$ a derivada em ponto $P_0$
é a melhor aproximação \emph{linear} de $f$, ela é uma aplicação linear $A: \R^n \to \R^m$
tal que $F(P) = F(P_0) + A (P-P_0) + G(P-P_0)$ e $\lim_{P\to P_0} \frac{|G(P-P_0)|}{|P-P_0|} = 0$.
Comparando as duas definições e tendo em visto que a aplicação $z \to az : \C \to \C$
é linear sobre $\R$, vejamos que existência de derivada complexa implica existência de derivada real.
\end{proof}

\begin{teorema}
Uma função $f(z)$ tem uma derivada complexa em ponto $z_0$ se e somente se
ela é diferenciável como $R^2\supset D \to \R^2$ e as derivadas parciais
satisfazem as equações de Cauchy--Riemann.
\end{teorema}

\begin{problema}
Suponha $f$ e $g$ têm as derivadas complexas em ponto $z_0$.
Provar que $f+g$, $f\cdot g$ tem as derivadas complexas em ponto $z_0$
e eles são iguais $(f+g)'(z_0) = f'(z_0)+g'(z_0)$ e $(fg)'(z_0) = f(z_0) g'(z_0) + f'(z_0) g(z_0)$.
\end{problema}

\begin{problema}
Suponha $f: D_0 \to D_1$ tem derivada complexa em $z_0$,
e $g : D_1 \to D_2$ tem derivada complexa em $f(z_0)$.
Provar que $g\circ f: D_0 \to D_2$ tem derivada complexa em $z_0$
e ela é igual a $(g\circ f)'(z_0) = g'(f(z_0)) \cdot f'(z_0)$. 
\end{problema}

\section{Funções holomorfas}

\subsection{Holomorficidade}

\begin{defin} Uma função é \emph{holomorfa} em região $D$ se ela tem derivada complexa em cada ponto de região.
Uma função é holomorfa em ponto, se ela é holomorfa em alguma vizinhança dele.
\end{defin}

\begin{exem}
Constantes f(z) = a,
identidade f(z) = z,
funções lineares $z \mapsto a z$,
todos os polinômios com coeficientes complexos.
\end{exem}

\begin{problema}
Provar que a função holomorfa com derivada não nula é uma aplicação conforme,
i.e. ela preserve todos os ângulos entre as (retas tangentes das) curvas.
\end{problema}


\subsection{Séries}

Todas as definições e teoremas sobre séries reais funcionam em caso complexo
se usar valor absoluto complexo em vez de valor absoluto real.

\begin{problema}
Suponha que $\sum_{n=0}^\infty a_n'(z)$ é uniformemente convergente.
Provar que se a série $\sum_{n=0}^\infty a_n(z)$ é convergente ao menos em um ponto,
então ela é convergente uniformemente e $(\sum_{n=0}^\infty a_n)' = \sum_{n=0}^\infty a_n'$.
\end{problema}

\begin{problema}[Fórmula de Cauchy--Hadamard]
Seja $R = \big(\limsup_{n\to\infty} |c_n|^{1/n}\big)^{-1}$. Provar que $f(z) = \sum_{n=0}^\infty c_n z^n$
converge absolutamente em disco $D := \{z\in\C: |z|<R\}$ e diverge na $\{z\in\C: |z| >R\}$.
Também provar que em qualquer compacto $K\subset D$ ele converge uniformemente.
\end{problema}

\begin{problema}
Provar que séries para exponente $\exp(z) = \sum \frac{z^n}{n!}$,
seno e cosseno convergem absolutamente no todo $\C$ e são holomorfas.

\end{problema}

\section{Integração complexa}

\begin{defin}
\end{defin}

\begin{exem}
\end{exem}

\begin{teorema}
\end{teorema}

\section{Teorema de integral de Goursat--Cauchy.}

\begin{remark}
Usando a fórmula de Green\footnote{
A fórmula de Green é um caso particular da fórmula de Stokes 
$$\int_{\partial D} \omega = \int_D d\omega$$
em qual $\omega = f(x,y) dx + g(x,y) dy$, e então
$d\omega = (-\frac{\partial f(x,y)}{\partial y} + \frac{\partial g(x,y)}{\partial x}) dx\wedge dy$.}
Cauchy provou que se $f(z)$ é uma função holomorfa em interior de uma região $D$
tal que a derivada $f'(z)$ é contínua, então $\int_{\partial D} f(z) dz = 0$.
Se $f(z) = u(x,y) + v(x,y) i$ então
$\int_{\partial D} f(z) dz = \int_{\partial D} (u(x,y) + i v(x,y)) (dx + i dy)
= \int_{\partial D} (u(x,y) dx - v(x,y) dy) + i \int_{\partial D} v(x,y) dx + u(x,y) dy =
= \int_D (-u_y-v_x) dx dy + i \int_D (-v_y + u_x) dx dy = \int_D 0 + i \int_D 0 = 0$,
onde usamos as equações de Cauchy--Riemann $u_x = v_y$ e $u_y = - v_x$.
Depois em 1883 Goursat removeu a condição de continuidade de derivada ---
a integral já é zero se somente suponhamos que $f$ é holomorfa
(tem derivada complexa) no interior de região $D$ e contínua na fronteira $\partial D$.
Nessa seção damos a demonstração essencialmente de Goursat.
\end{remark}

\begin{lema}
Seja $f$ uma função continua no triângulo $\Delta$ e holomorfa no interior dele.
Então $\int_{\partial\Delta} f(z) dz = 0$.
\end{lema}
\begin{proof}
A demonstração segue a ideia de Goursat (1883).
Suponha $|\int_{\partial\Delta} f(z) dz| = M > 0$.
Decompomos um triangulo em $4$ triângulos de meio tamanho\footnote{
Se $\Delta = ABC$ e $AD$, $BE$, $CF$ são as três medianas do triangulo,
considerar os triângulos $DFE$, $AFD$, $FBE$, $ECD$.}
Para um dos triângulos (chama ele $\Delta_1$) temos
$|\int_{\partial\Delta} f(z) dz| \geq \frac{M}{4}$.
Similarmente construímos $\Delta = \Delta_0 \supset \Delta_1 \supset \Delta_2 \supset ...$.
Seja $z_0$ um ponto no interseção de todos $\Delta_k$.\footnote{
Este ponto é único desde os diâmetros de triângulos vai ao zero,
e existe e.g. toma o limite dos centros, ou usa compacidade.}
Temos $f(z) - f(z_0) - f'(z_0) (z-z_0) = \alpha(z) (z-z_0)$ t.q.
$\lim_{z\to z_0} \alpha(z) = 0$. Então $\forall \epsilon \exists \delta$ tal que
$|z-z_0|<\delta \implies |\alpha(z)|<\epsilon$. Escolhemos $k$ tal que
$\Delta_k$ é dentro de $\{z: |z-z_0|<\delta\}$.
Temos
$$\int_{\Delta_k} f(z) dz = (f(z_0)-f'(z_0)z_0) \int_{\Delta_k} dz + f'(z_0) \int_{\Delta_k} d\frac{z^2}{2}
+ \int_{\Delta_k} \alpha(z) (z-z_0) dz = 0 + 0 + \int_{\Delta_k} \alpha(z) (z-z_0) dz \leq \epsilon \cdot ...$$
\end{proof}

\begin{teorema}[Cauchy--Goursat 1883]
Suponha que $C$ é uma curva fechada de cumprimento finito t.q. $C = \partial D$
e uma função $f(z)$ é continua na $C$ e holomorfa no interior de $D$. Então
$\int_C f(z) dz = 0$.
\end{teorema}
\begin{proof}
Se $C$ é linear em pedaços, pela indução decompomos ela em soma de fronteiras orientadas de triângulos.
Outras curvas de cumprimento finito aproximamos por curvas lineares em pedaços.
\end{proof}

\section{Primitiva.}


\section{Fórmula integral de Cauchy.}

\section{Decomposição em séries de Taylor.}

\section{Teorema de Morera. Critério de holomorficidade.}

\section{Teorema de Weierstrass.}

\section{Funções holomorfas em um anel. Séries de Laurent.}

\section{Pontos singulares isolados.}

\section{Resíduos. Valor principal de Cauchy.}

\section{Princípio do argumento.}
\begin{defin}
\end{defin}
\begin{teorema}
\end{teorema}
\begin{teorema}[Rouché]
\end{teorema}
\begin{cor}[Teorema Fundamental de Álgebra]
\end{cor}

\section{Propriedades topológicas das funções meromorfas}

\section{Teorema de Montel. Funcionais contínuos nas famílias compactas de funções.}

\section{Teorema de Hurwitz. Funções univalentes.}

\section{Continuação analítica.}

\section{Teorema de Riemann.}

\section{Automorfismos de domínios simplesmente conexos.}

\section{Superfícies de Riemann. Uniformização.}

\begin{teorema}[Poincaré, Koebe 1907, foi 22o problema de Hilbert]
Qualquer superfície de Riemann simplesmente conexa é biholomorfa ao uma de três superfícies padrão:
esfera de Riemann $\C\P^1 = \bar{C}$, reta/plano complexo $\C$ ou disco $\D = \{z : |z|<1\}$.
\end{teorema}
Os métodos de demonstração deste teorema são fora deste curso - temos que construir alguns funções
harmônicas, e para isso usamos EDP. Veja e.g. livro texto de Donaldson "Riemann Surfaces" (2011).

\section{Grupos de Fuchs.}

\begin{exem}[Grupo modular]
$\Gamma = \{ z \mapsto \frac{az+b}{cz+d} \,|\, a,b,c,d\in\Z, \, ad-bc>0 \} = PSL(2,\Z)$
\end{exem}

\section{Espaço de módulos de toros complexos.}

\begin{teorema}
O espaço de módulos de toros complexos é identificado com o quociente $\H/\Gamma$
do plano superior por grupo modular.
\end{teorema}

\section{Funções analíticas.}

\section{Exemplos das superfícies de Riemann das funções analíticas.}
