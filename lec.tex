
\section*{Números complexos e plano euclidiano}

Lembranças sobre os números complexos antes de começar a análise.

Números complexos $\C$ são munidos com as operações binárias
de soma e produto $+,\cdot : \C \times \C \to \C$,
e com as funções de 
conjugação $\ol{\cdot} : \C \to \C$,
parte real e parte imaginária $\Re,\Im : \C \to \R$,
norma e valor absoluto $N,|\cdot| : \C \to \R_{\geq 0}$,
argumento $\arg : \C \to \R/2\pi\Z$,
valor principal de argumento $Arg : \C \to [0,2\pi)$.
\marginnote{
\begin{gather*}
z = x + y i \\
z = r (\cos \phi + i \sin \phi) = r e^{i\phi}\\
\ol{z} = x - y i = r e^{-i\phi} \\
x = r \cos \phi \\
y = r \sin \phi \\
N(z) = z \ol{z} = r^2 = x^2+y^2 \\
|z| = \sqrt{N(z)} = r \\
\arg z = \phi \mod 2\pi
\end{gather*}
}

A norma é multiplicativa:
$$ N(zw) = N(z) N(w) $$

Podemos munir $\C$ com a função de \emph{distância} $d(z,w) := |z-w|$.
Ele é uma métrica\footnote{Isso é: $d(z,w) \geq 0$, $d(z,w) = 0$ se e só se $z=w$,
$d(z,w) = d(w,z)$ e $d(z,w) \leq d(z,s) + d(s,w)$ (desigualdade de triângulo).}
Pelo teorema de Pitágoras essa métrica coincide com a métrica euclidiana.

Similarmente podemos munir $\C$ com uma \emph{orientação}, declarando que o vetor $i v$ é no lado esquerda de vetor $v$.

Multiplicatividade de norma implica que $d(az+b,aw+b) = |a| d(z,w)$, i.e. se $|a|=1$ a transformação $z\mapsto az+b$
é uma isometria (preserva as distâncias) euclidiana.

Então geometricamente (Caspar Wessel 1799, e depois Gauss e Argand) 
podemos identificar $\C$ com $\R^2$ e com um plano euclidiano.
As transformações $z \to z + b$ são translações,
as transformações $z \to (\cos \phi + i \sin \phi) \cdot z$ são rotações com centro em $0$,
as transformações $z \to r \cdot z$ ($r\in\R_{>0}$) são dilatações. 
Todas elas preservam a orientação e também são conformes, i.e. preservam os ângulos.
Também todas $z \to (\cos \phi + i \sin \phi) \cdot z + b$ preservam as distâncias euclidianas.
As transformações $z \to (\cos \phi + i \sin \phi) \cdot \ol{z} + b$ coincidem com as reflexões
em diferentes retas, elas preservam as distâncias, mas trocam a orientação.

O subconjunto $U \subset \C$ é chamado \emph{aberto}\footnote{com respeito de topologia euclidiana}
se ele é uma união de bolas abertas $B(z,r) := \{w\in \C t.q. |w-z|<r\}$ ($z\in\C,r\in\R$), 
i.e. $\forall z\in U \exists r>0 \textit{tal que} \forall w\in\C |w-z|<r \implies w\in U$,
em palavras -- qualquer ponto de $U$ também tem uma vizinhança dele dentro de $U$.

Na análise complexa as palavras \emph{região} ou \emph{domínio} significam um subconjunto aberto de $\C$.
Também em seguida geralmente temos em mente subconjuntos \emph{conexos}, i.e. tais
que não são uniões disjuntas de dois subconjuntos abertos não vazios.
O propriedade que será mais útil para nós é \emph{conexidade em caminhos}:
para qualquer par de pontos $P,Q\in U$ existe um caminho em $U$ entre eles,
i.e. existe uma função contínua $\gamma: [0,1] \to U$ tal que $\gamma(0) = P$
e $\gamma(1)=Q$. Se quiser pode se imaginar que essa função é linear em pedaços,
ou ainda feita só de intervalos horizontais e verticais.

Muitas definições e teoremas básicas sobre as funções de variável real
podem ser promovidas para as funções de variável complexa usando
a interpretação de $|z|$ acima.

\section{Diferenciação complexa}

\subsection{Funções de variável complexa}

Sejam $D\subset \C$ uma região e $f : D \to \C$ uma função,
$D\supset z = x + yi$, $f(z) = w = u + v i$ para $x,y,u,v\in \R$, $z\in D\subset \C$, $w\in \C$.
Podemos também interpretar $f$ como uma função de 1 variável complexa com um valor complexo,
ou como uma função $u(x,y) + v(x,y) i : \R^2\supset D \to \C$ de $2$ variáveis reais com um valor complexo,
ou como um par de funções de $2$ variáveis reais
$u,v : D \to \R$, $u(x,y), v(x,y) \in \R$,
i.e. $(x,y) \mapsto (u(x,y),v(x,y)) : \R^2 \supset D \to \R^2$.

Para $z$ complexo o limite $\lim_{z\to 0}$ significa $\lim_{|z|\to 0}$.
Similarmente $\lim_{z\to z'}$ é mesmo que $\lim_{|z-z'|\to 0}$.

\subsection{Derivada complexa}
Em seguida definição $\cdot$ é multiplicação em $\C$,
geralmente omitimos este símbolo, mas quero enfatizar que essa definição depende
de multiplicação (resp. divisão) de números complexos.
\begin{defin}
Suponha que existe $a\in \C$ tal que
$$ f(z) = f(z_0) + a \cdot (z-z_0) + g(z) \cdot (z-z_0) $$
tal que $\lim_{z\to z_0} g(z) = 0$,
i.e. $\lim_{z\to z_0} \frac{f(z)-f(z_0)}{z-z_0} = a$.
Neste caso digamos que $f$ tem a derivada complexa em $z_0$ igual ao $a\in \C$
e denotamos $f'(z_0) := a$, alternativamente usamos notação $\frac{df}{dz}(z_0) = f'(z_0)$.
Em linguagem $\epsilon-\delta$:
$\forall\epsilon>0\exists\delta>0$ t.q. $|z-z_0|<\delta$ implica
$|f(z)-f(z_0)-a(z-z_0)| < \epsilon |z-z_0|$.
\end{defin}

\begin{lema}
Se $f(z) = u(x,y) + iv(x,y)$ tem a derivada complexa em ponto $z_0$ neste ponto temos as equações de Cauchy--Riemann
para derivadas parciais: 
\begin{equation}
u_x = v_y, \,\,\, u_y = - v_x
\end{equation}
onde $u_x = \frac{\partial u(x,y)}{\partial x} = \lim_{\epsilon\to 0} \frac{u(x+\epsilon,y)-u(x,y)}{\epsilon}$,etc
são as derivadas parciais.
\end{lema}

\emph{Diferenciais complexas} (formas diferenciais complexas).
Em notação-definição $\frac{df}{dz}(z_0) = f'(z_0)$ podemos formalmente multiplicar pelo denominador para obter
uma equação $(df)(z_0) = f'(z_0) (dz)(z_0)$.
Por enquanto podemos formalmente definir um espaço vetorial complexo $1$-dimensional com um vetor-base $(dz)(z_0)$ e para cada função $f(z)$ com derivada complexa em ponto $z_0$ associar outro vetor $(df)(z_0) := f'(z_0) (dz)(z_0)$. Essa notação (de Leibniz)
será útil ao menos para escrever as fórmulas mais convenientes.

Notações $\partial_z$ e $\partial_{\ol{z}}$.
Equações de Cauchy--Riemann são equivalentes ao equação $\partial_{\ol{z}} f(z_0) = 0$,
neste caso $\partial_z f(z_0) = f'(z_0)$.

\begin{lema}
Se $f$ tem derivada complexa $f'(z_0) = r (\cos\phi + i \sin \phi)$ em ponto $z_0$
ela também tem derivada como aplicação
de $D \subset \R^2$ para $R^2$, e ela é igual a
$\begin{smallmatrix} u_x & -v_y & u_y & v_x \end{smallmatrix}
= r \begin{smallmatrix} \cos\phi & -\sin\phi & \sin\phi & \cos\phi \end{smallmatrix} $.
\end{lema}
\begin{proof}
Lembra que geralmente para aplicação $F: \R^n \to \R^m$ a derivada em ponto $P_0$
é a melhor aproximação \emph{linear} de $f$, ela é uma aplicação linear $A: \R^n \to \R^m$
tal que $F(P) = F(P_0) + A (P-P_0) + G(P-P_0)$ e $\lim_{P\to P_0} \frac{|G(P-P_0)|}{|P-P_0|} = 0$.
Comparando as duas definições e tendo em visto que a aplicação $z \to az : \C \to \C$
é linear sobre $\R$, vejamos que existência de derivada complexa implica existência de derivada real.
\end{proof}

\begin{teorema}
Uma função $f(z)$ tem uma derivada complexa em ponto $z_0$ se e somente se
ela é diferenciável como $R^2\supset D \to \R^2$ e as derivadas parciais
satisfazem as equações de Cauchy--Riemann.
\end{teorema}

\begin{problema}
Suponha $f$ e $g$ têm as derivadas complexas em ponto $z_0$.
Provar que $f+g$, $f\cdot g$ tem as derivadas complexas em ponto $z_0$
e elas são
\begin{gather}
(f+g)'(z_0) = f'(z_0)+g'(z_0), \\
(fg)'(z_0) = f(z_0) g'(z_0) + f'(z_0) g(z_0).
\end{gather}
\end{problema}

\begin{problema}
Suponha $f: D_0 \to D_1$ tem derivada complexa em $z_0$,
e $g : D_1 \to D_2$ tem derivada complexa em $f(z_0)$.
Provar que $g\circ f: D_0 \to D_2$ tem derivada complexa em $z_0$
e
\begin{equation}
(g\circ f)'(z_0) = g'(f(z_0)) \cdot f'(z_0).
\end{equation} 
\end{problema}

\section{Funções holomorfas}

\subsection{Holomorficidade}

\begin{defin} Uma função é \emph{holomorfa} em região $D$ se ela tem derivada complexa em cada ponto de região.
Uma função é holomorfa em ponto, se ela é holomorfa em alguma vizinhança dele.
\end{defin}

\begin{exem}
Constantes f(z) = a,
identidade f(z) = z,
funções lineares $z \mapsto a z$,
todos os polinômios com coeficientes complexos $f \in \C[z]$
são holomorfas no todo $\C$ (tais funções são chamadas inteiras).
\end{exem}

\begin{problema}
Provar que se $f'(z)=0$ para todo $z$ em região $D\subset \C$, então a função $f$ é constante na $D$.
(NB: você usou que $D$ é conexa?)
\end{problema}

\begin{problema}
Provar que a função holomorfa com derivada não nula é uma aplicação conforme,
i.e. ela preserve todos os ângulos entre as (retas tangentes das) curvas.
\footnote{
Dica: primeiro prove que $z \mapsto b+ a z$ é a composição de rotação por $\arg z$, dilatação em $|r|$
e translação por $b$, e então é conforme.
Depois para $z \mapsto f(z)$ com $f(z) = 0$ e $f'(z)=1$ verifique que o ângulo entre um raio e a imagem dele
é igual ao $0$ (se não saber a definição, use $\lim_{z\to 0} ang(z,0,f(z)) = \lim_{z\to 0} \arg \frac{f(z)}{z}$).
}
\end{problema}

Se $f'(0)=0$, o ângulo não é preservado, mas multiplicado por ordem de zero. Vamos provar (mas ainda não provamos) que $f$ tem todas as derivadas. Para agora, verifique que $z\mapsto z^n$ multiplica os ângulos em $z=0$ em $n$ vezes
(mas preserva os ângulos em outros pontos).

\subsection{Séries}

Todas as definições e teoremas sobre séries reais funcionam em caso complexo
se usar valor absoluto complexo em vez de valor absoluto real:
em vez de intervalos $\{x\in \R\,|\,|x-x_0|<r\}$ escreveremos $\{z\in \C\,|\,|z-z_0|<r\}$, etc.

\begin{problema}
Suponha que $\sum_{n=0}^\infty a_n'(z)$ é uniformemente convergente.
Provar que se a série $\sum_{n=0}^\infty a_n(z)$ é convergente ao menos em um ponto,
então ela é convergente uniformemente e $(\sum_{n=0}^\infty a_n)' = \sum_{n=0}^\infty a_n'$.
\end{problema}

\begin{problema}[Fórmula de Cauchy--Hadamard]
Seja $R = \big(\limsup_{n\to\infty} |c_n|^{1/n}\big)^{-1}$. Provar que $f(z) = \sum_{n=0}^\infty c_n z^n$
converge absolutamente em disco $D := \{z\in\C: |z|<R\}$ e diverge na $\{z\in\C: |z| >R\}$.
Também provar que em qualquer compacto $K\subset D$ ele converge uniformemente.
\marginnote{Usa progressão geométrica para provar convergência.
Para divergência prove que se séries $\sum c_n$ então converge o limite dos termos $c_n$ é igual ao $0$.}
\end{problema}

\begin{problema}
Provar que séries para exponente, seno e cosseno
\begin{align}
\exp(z) &= \sum_{n=0}^\infty \frac{z^n}{n!}, \\
\cos(z) &= \sum_{n=0}^\infty (-1)^n \frac{z^{2n}}{(2n)!}     &= \frac{\exp(iz) + \exp(-iz)}{2}, \\
\sin(z) &= \sum_{n=0}^\infty (-1)^n \frac{z^{2n+1}}{(2n+1)!} &= \frac{\exp(iz) - \exp(-iz)}{2i},
\end{align}
convergem absolutamente no todo $\C$ e são holomorfas.
Compute as derivadas delas:
$\exp' = \exp$, $\cos' = -\sin$, $\sin' = \cos$.
\end{problema}

As séries $\sum a_n z^n$ e $\sum \ol{a_n} w^n$ tem o mesmo raio de convergência e por qualquer $z$ com a série convergente temos
$\overline{\sum a_n z^n} = \sum \ol{a_n} \cdot \ol{z}^n$.

\begin{prop}
Para todos $z,w\in \C$ temos
\begin{equation}
\exp(z+w) = \exp(z) \exp(w)
\end{equation}
\end{prop}
\begin{proof}
Para $w$ fixo a função em $z$ dada por
$f(z) := \exp(z) \exp(w-z)$ é holomorfa (como o produto das composições de funções holomorfas).
A derivada dela é igual $f'(z) = \exp'(z) \exp(w-z) + \exp(z) \exp'(w-z) = \exp(z) \exp(w-z) - \exp(z) \exp(w-z) = 0$
(pela regra de Leibniz e regra de cadeias). Logo ela é constante e igual $f(0) = \exp(z)$.
\end{proof}

\begin{cor}[Fórmula de Euler]
\begin{equation}
\exp(u+iz) = \exp(u) \big( \cos(z) + i \sin(z) \big)
\end{equation}
\end{cor}
\begin{proof}
Se você já provou a relação entre funções $\cos$ e $\sin$ de origem geométrico e as séries acima,
o corolário é óbvio. Senão agora podemos diretamente relacionar: veja que a derivada de
$(\cos(\phi) + i \sin(\phi)) \exp(-i\phi)$ (considerada como função de argumento real e valor complexo
onde as funções trigonométricas $\cos$ e $\sin$ são definidas geometricamente)
existe e é igual $0$.
\footnote{
Outra demonstração mais geométrica é seguinte:
desde que a série $\exp(z) = \sum z^n / n!$ tem todos os coeficientes reais e $-i = \ol{i}$ temos
$\exp(-i\phi) = \sum (-i)^n \phi^n = \overline{\exp(i\phi)}$ por qualquer $\phi \in \R$.
Logo $|\exp(i\phi)|^2 = \exp(i\phi) \overline{\exp(i\phi)} = \exp(i\phi) \exp(-i\phi) = 1$.
Logo $\exp(i\phi) = \cos_{geom}(\psi) + i \sin_{geom}(\psi)$ onde $\psi = \arg \exp(i\phi)$.
Para qualquer número complexo $z$ a parte real dele é igual $\frac{z+\ol{z}}2$
e a parte imaginária é $\frac{z-\ol{z}}2$, logo $\cos_{série}(\phi) = \frac{\exp(i\phi)+\exp(-i\phi)}2 = \Re \exp(i\phi) = \cos_{geom}(\psi)$ e similarmente para $\sin$.
Ex - finish this proof.
}
\end{proof}

\begin{remark}
Acima são os casos do método geral:
se as duas funções satisfazem a mesma equação diferencial ordinária sem singularidades,
e tem as mesmas condições iniciais, elas são iguais.
\end{remark}

\begin{cor}[Fórmulas trigonométricas]
\begin{multline}
\cos(z+w) = \cos(z)\cos(w) - \sin(z)\sin(w) \\
\sin(z+w) = \cos(z)\sin(w) - \sin(z)\cos(w) \\
\end{multline}
\end{cor}

\section{Integração complexa}

Se em soma integral $S = \sum f(\xi_k) \Delta z_k$ a variável e o valor são complexos
obtemos uma definição de integral de Riemann de função sobre uma curva $\gamma \subset \C$
(ou $\gamma \to \C$, não necessariamente injetora). Temos

$$ S = \sum (u(\xi_k)+v(\xi_k)i) \Delta (x_k + y_k i) 
= \sum (u(\xi_k) \Delta x_k - v(\xi_k) \Delta y_k) + (u(\xi_k) \Delta y_k + v(\xi_k) \Delta x_k)i $$

\begin{defin}
A integral de $f(z) = u(x,y) + v(x,y) i$ sobre a curva orientada $\gamma \subset D \subset \C$ é um número complexo
\[ \int_\gamma f(z) dz := \int_\gamma (u dx - v dy) + i \int_\gamma (v dx + u dy) \]
\end{defin}

Se $w: [a,b] \to D$ é uma parametrização suave em pedaços da curva $\gamma \subset D$
dada por $w(t) = x(t) + y(t) i$ temos
\begin{equation}
\int_\gamma f(z) dz = 
\int_a^b (u(w(t)x'(t)-v(w(t))y'(t))dt + i \int_a^b (u(w(t)y'(t)+v(w(t))x'(t))dt
= \int_a^b f(w(t)) w'(t) dt,
\end{equation}
em particular o valor do lado direito não depende da parametrização $w$.

\begin{exem}
Compute $\int_{|z-z_0|=r} (z-z_0)^n dz$ por qualquer $z_0\in\C$, $r\in\R$ e $n \in \Z$.
$$ \int_{|z|=1} zˆn dz = \int_0^{2\pi} \exp(i(n+1)\phi) d\phi = 2\pi i \delta_{n,-1} $$
i.e. a integral é $0$ para $n\neq -1$ e é $2\pi i$ para $n=-1$.
\end{exem}

\begin{teorema}
A aplicação $\int_\gamma \cdot dz : f \mapsto \int_\gamma f(z) dz$ é $\C$-linear em $f$:
$$ \int_\gamma (a f + b g) dz = a \int_\gamma f(z) dz + b \int_\gamma g(z) dz $$
e também linear em $\gamma$:
$$ \int_{\gamma_1 \cup \gamma_2} f(z) dz = \int_{\gamma_1} f(z) dz + \int_{\gamma_2} f(z) dz.$$
Aqui pensamos em $\gamma_i$ orientadas tais que o fim de $\gamma_1$ é o início de $\gamma_2$,
mas também podemos estender a definição de integral sobre as cadeias (somas formais de curvas orientadas).
\end{teorema}

\section{Teorema de integral de Goursat--Cauchy.}

\begin{remark}
Usando a fórmula de Green\footnote{
A fórmula de Green é um caso particular da fórmula de Stokes 
$$\int_{\partial D} \omega = \int_D d\omega$$
em qual $\omega = f(x,y) dx + g(x,y) dy$, e então
$d\omega = (-\frac{\partial f(x,y)}{\partial y} + \frac{\partial g(x,y)}{\partial x}) dx\wedge dy$.}
Cauchy provou que se $f(z)$ é uma função holomorfa em interior de uma região $D$
tal que a derivada $f'(z)$ é contínua, então $\int_{\partial D} f(z) dz = 0$.
Se $f(z) = u(x,y) + v(x,y) i$ então
\begin{multline*}
\int_{\partial D} f(z) dz = \int_{\partial D} (u(x,y) + i v(x,y)) (dx + i dy) \\
= \int_{\partial D} (u(x,y) dx - v(x,y) dy) + i \int_{\partial D} v(x,y) dx + u(x,y) dy \\
= \int_D (-u_y-v_x) dx dy + i \int_D (-v_y + u_x) dx dy = \int_D 0 + i \int_D 0 = 0,
\end{multline*}
onde usamos as equações de Cauchy--Riemann $u_x = v_y$ e $u_y = - v_x$.
Depois em 1883 Goursat removeu a condição de continuidade de derivada ---
a integral já é zero se somente suponhamos que $f$ é holomorfa
(tem derivada complexa) no interior de região $D$ e contínua na fronteira $\partial D$.
Nessa seção damos a demonstração essencialmente de Goursat.
\end{remark}

\begin{lema}
Seja $f$ uma função continua no triângulo $\Delta$ e holomorfa no interior dele.
Então $\int_{\partial\Delta} f(z) dz = 0$.
\end{lema}
\begin{proof}
A demonstração segue a ideia de Goursat (1883).
Suponha $|\int_{\partial\Delta} f(z) dz| = M > 0$.
Decompomos um triangulo em $4$ triângulos de meio tamanho\footnote{
Se $\Delta = ABC$ e $AD$, $BE$, $CF$ são as três medianas do triangulo,
considerar os triângulos $DFE$, $AFD$, $FBE$, $ECD$.}
Para um dos triângulos (chama ele $\Delta_1$) temos
$|\int_{\partial\Delta} f(z) dz| \geq \frac{M}{4}$.
Similarmente construímos $\Delta = \Delta_0 \supset \Delta_1 \supset \Delta_2 \supset ...$.
Seja $z_0$ um ponto no interseção de todos $\Delta_k$.\footnote{
Este ponto é único desde os diâmetros de triângulos vai ao zero,
e existe e.g. toma o limite dos centros, ou usa compacidade.}
Temos $f(z) - f(z_0) - f'(z_0) (z-z_0) = \alpha(z) (z-z_0)$ t.q.
$\lim_{z\to z_0} \alpha(z) = 0$. Então $\forall \epsilon \exists \delta$ tal que
$|z-z_0|<\delta \implies |\alpha(z)|<\epsilon$. Escolhemos $k$ tal que
$\Delta_k$ é dentro de $\{z: |z-z_0|<\delta\}$.
Temos
\begin{multline}
\int_{\Delta_k} f(z) dz = \\
(f(z_0)-f'(z_0)z_0) \int_{\Delta_k} dz + f'(z_0) \int_{\Delta_k} d\frac{z^2}{2}
+ \int_{\Delta_k} \alpha(z) (z-z_0) dz \\
= 0 + 0 + \int_{\Delta_k} \alpha(z) (z-z_0) dz = \int_{\Delta_k} \alpha(z) (z-z_0) dz,
\end{multline}
Mas o último integral é pequeno: $|z-z_0|$ é limitado por diametro de $\Delta_k$,
que é $C_1 2^{-k}$, e $\int_{\Delta_k} |dz|$ é o perimetro de $\Delta_k$, também igual a $C_2 2^{-k}$,
onde $C_1,C_2$ são constantes absolutas
logo
\begin{multline*}
M 4^{-k} \leq |\int_{\Delta_k} f(z) dz| = |\int_{\Delta_k} \alpha(z) (z-z_0) dz| \\
\leq \int_{\Delta_k} |\alpha(z)| |z-z_0| |dz| \leq \epsilon C_1 C_2 4^{-k},
\end{multline*}
então $M \leq C_1 C_2 \epsilon$ por qualquer $\epsilon>0$, i.e. $M=0$.
\end{proof}

\begin{teorema}[Cauchy--Goursat 1883]
\label{t:cauchy}
Suponha que $C$ é uma curva fechada de cumprimento finito t.q. $C = \partial D$
e uma função $f(z)$ é continua na $C$ e holomorfa no interior de $D$. Então
$\int_C f(z) dz = 0$.
\end{teorema}
\begin{proof}
Se $C$ é linear em pedaços, pela indução decompomos ela em soma de fronteiras orientadas de triângulos.
Outras curvas de cumprimento finito aproximamos por curvas lineares em pedaços.
\end{proof}

\begin{teorema}
\label{t:cauchy-k}
Suponha que $f$ é holomorfa em $D\subset \C$ e $K \subset D$ é um compacto t.q.
a fronteira dele $\partial K$ é finito número de curvas fechadas. Então $\int_{\partial K} f(z) dz = 0$,
onde as orientações das componentes da fronteira são induzidas pela orientação de $\C$ e $K$.
\end{teorema}



\subsection{Cadeias, fronteiras,  cíclos e número de Betti}

Mais geralmente podemos definir os grupos de \emph{cadeias} ($0$-cadeias, $1$-cadeias e $2$-cadeias)
como os grupos de combinações lineares finitas com coeficientes inteiros de pontos, intervalos orientados e triângulos em $D$.

Formalmente, o grupo de $2$-cadeias $C_2(D)$ é o conjunto de todas as somas finitas formais $\sum_{k=1}^n c_k \Delta_k$ de triângulos,
o grupo de $1$-cadeias $C_1(D)$ é o conjunto de todas somas formais $\sum_{k=1}^n b_k P_kQ_k$ de arestas orientadas,
e o grupo de $0$-cadeias $C_0(D)$ é o conjunto de todas somas formais $\sum a_k P_k$ de pontos
com coeficientes $a_k,b_k,c_k \in \Z$ inteiros. Soma e diferença de cadeias é definida formalmente, e.g.
\[ \sum c_k \Delta_k - \sum c_k' \Delta_k = \sum (c_k - c_k') \Delta_k. \]

Pela qualquer função $f(z)$ podemos definir a ``integral'' (ou o ``valor'') dela no $0$-cadeia pela linearidade
\[ \int_{\sum a_k P_k} f := \sum a_k f(P_k). \]

Similarmente pelo qualquer diferencial $f(z) dz$ podemos definir a integral dele sobre $1$-cadeia pela linearidade
\[ \int_{\sum b_k P_kQ_k} f(z) dz := \sum b_k \int_{P_kQ_k} f(z) dz \]
onde 
\[ \int_{P_k Q_k} f(z) dz := \int_0^1 f(P_k + t(Q_k-P_k)) (Q_k-P_k) dt. \]

Podemos definir aplicação linear de \emph{fronteira}
\[ \partial: C_2(D) \to C_1(D) \]
pelo
\[ \partial \sum c_k \Delta_k = \sum c_k \partial \Delta_k, \]
onde a fronteira de um triangulo orientado $ABC$ é definida como 
\[ \partial ABC := AB + BC + CA, \]
i.e. a fronteira de triângulo é a soma de 3 arestas dele orientadas em sentido anti-horário.

Então como corolário do lema de Goursat temos que
\[ \int_{\partial S} f(z) dz = 0 \]
por qualquer $2$-cadeia $S \in C_2(D)$.

Também podemos definir aplicação de fronteira 
\[ \partial : C_1(D) \to C_0(D) \]
pela
\[ \partial \sum b_k P_k Q_k = \sum b_k \partial  P_kQ_k = \sum b_k (Q_k - P_k). \]

Uma $1$-cadeia $C = \sum b_k P_k Q_k$ é chamada \emph{$1$-cíclo} se a fronteira dela é zero, i.e. $\partial C = 0$.

Para um triângulo $\Delta = ABC$ temos
\[
\partial \Delta = AB + BC + CA,
\]
então a fronteira de fronteira é igual ao zero:
\[
\partial \partial \Delta = \partial (AB + BC + CA)  = \partial AB + \partial BC + \partial CA = B-A + C-B + A-C = 0.
\]

Logo pela linearidade também temos que por qualquer $2$-cadeia $S = \sum c_k \Delta_k$
\[ \partial \partial S = \sum c_k \partial \partial \Delta_k = \sum c_k \cdot 0 =  0. \]

Isso implica que qualquer $1$-fronteira ($1$-cadeira que é uma fronteira de uma $2$-cadeia) é um $1$-cíclo.

As vezes você pode encontrar as notações $Z_1(D)$ para o grupo de $1$-cíclos
\[ Z_1(D) := \{ C \in C_1(D) \,|\, \partial C = 0 \} \subset C_1(D), \]
e para o grupo de $1$-fronteiras
\[ B_1(D) := \partial C_2(D) \subset C_1(D) \].
Vejamos que há uma inclusão natural $B_1(D) \subset Z_1(D) \subset C_1(D)$,
i.e. cada fronteira é um cíclo.

Por algumas regiões acontece que todos os $1$-cíclos são $1$-fronteiras,
por exemplo isso é o caso pelas regiões convexas.
Similarmente podemos definir regiões \emph{estreladas} como regiões tais que existe um ponto $z_0 \in D$ tal que
$\forall z\in D, \forall t \in [0,1] \implies z_0 + t (z-z_0) \in D$. Nas regiões estreladas também
todos os $1$-cíclos são $1$-fronteiras.
\marginnote{Mais geralmente isso é verdade em regiões \emph{simplesmente conexas},
i.e. regiões em quais qualquer par de caminhos entre par de pontos $P$ e $Q$ pode ser deformados um doutro continuamente com fins fixos,
i.e. $\forall w_0,w_1 : [0,1] \to D$ tais que $w_0(0) = w_1(0) = P$
e $w_0(1) = w_1(1) = Q$ existe $w: [0,1]\times[0,1] \to D$ tal que
$w(0,u) = w_0(u)$, $w(1,u) = w_1(u)$, $w(*,0) = P$, $w(*,1) = Q$, onde $w(*,0) = ...$ significa que valor de $*$ pode ser qualquer $t\in[0,1]$.}

Por algumas outras regiões isso não é verdade, e.g. em disco perfurado ou num anel há mais cíclos de que fronteiras.


Geralmente podemos medir essa diferença por grupo quociente
\[ H_1(D) := Z_1(D) / B_1(D), \] 
chamado o grupo de \emph{homologia}.
Os elementos deste grupo são as classes de equivalência de $1$-cíclos:
digamos que par de $1$-cíclos $C$ e $C'$ são equivalentes se a diferença deles é uma $1$-fronteira,
i.e. $C \simeq C' \iff \exists S \,\|\, C-C' = \partial S$.

O fato importante que podemos computar este grupo, por exemplo $H_1(anel) = \Z$.
Número de geradores dele é chamado \emph{número de Betti} $b_1(D)$.
Para $D$ o disco com $b$ buracos e furações o número de Betti é igual ao $b$,
i.e. ao número de buracos e furações. Isso significa que se somos interessados em integrais de formas holomorfas sobre
diferentes cíclos em $D$ temos que computar $b$ integrais independentes, e todos outros
já podem ser escritas como combinações lineares (com coeficientes inteiros) de $b$ integrais básicos
\[ \int_{a_1 C_1 + \dots + a_b C_b + \partial S} \omega = a_1 \int_{C_1} \omega + \dots + a_b \int_{C_b} \omega. \]
Uma escolha padrão dessa base $C_1,\dots,C_b$ é escolher os cíclos que atravesam cada buraco em sentido antihorário com uma volta.

\section{Primitiva.}


\section{Fórmula integral de Cauchy.}

\begin{teorema}
Seja $f : D \to \C$ holomorfa em região $D$
e $G = \{ z\in\C \,\|\, |z-z_0| \leq r \} \subset D$. Então
\[ f(z_0) = \frac1{2\pi i} \int_{\partial G} \frac{f(z)dz}{z-z_0} = \frac1{2\pi} \int_0^{2\pi} f(z_0 + r e^{i\phi}) d\phi. \]
\end{teorema}
\begin{proof}
Pela compacidade de $G$ temos que $f$ é uniformemente contínua em $G$, i.e.
$\forall \epsilon > 0 \exists\delta >0 \,:\, \forall |z-z_0|\leq\delta \implies |f(z)-f(z_0)| < \epsilon$.
Escolhemos um subdisco $G_\rho = \{ z\in\C \,\|\, |z-z_0| \leq \rho \} \subset G \subset D$.
Pelo \Cref{t:cauchy} (de Cauchy) temos
$\int_{\partial G} \frac{f(z) dz}{z-z_0} = \int_{\partial G_\rho} \frac{f(z) dz}{z-z_0}$.
Então
\begin{equation*}
\frac1{2\pi i} \int_{\partial G} \frac{f(z)dz)}{z-z_0} - f(z_0) = \frac1{2\pi i} \int_{\partial G_\rho} \frac{f(z)-f(z_0)}{z-z_0} dz, 
\end{equation*}
e 
\[|\frac1{2\pi i} \int_{\partial G_\rho} \frac{f(z)-f(z_0)}{z-z_0} dz| \leq \frac1{2\pi} \int_{\partial G_\rho} \frac{\epsilon}{\rho} |dz| = \epsilon. \]
Substituindo $z = z_0 + r e^{i\phi}$ obtemos
\[ f(z_0) = \frac{1}{2\pi i} \int_{\partial G} \frac{f(z)dz}{z-z_0} = 
\frac{1}{2\pi i} \int_0^{2\pi} \frac{f(z_0 + r e^{i\phi})}{r e^{i\phi}} \frac{d(z_0+re^{i\phi})}{d\phi} d\phi
= \frac1{2\pi} \int_0^{2\pi} f(z_0 + r e^{i\phi}) d\phi. \]
\end{proof}

\begin{teorema}[Cauchy]
Se função $f$ é holomorfa em região $D$ e $G\subset D$ um compacto t.q. a fronteira dele $\partial G$ é o finito número de curvas retificáveis
\footnote{E.g. poligonal ou arcos de círculos.},
então por qualquer $z_0 \in D \setminus \partial G$
$\frac1{2\pi i} \int_{\partial G} \frac{f(z)dz}{z-z_0}$  é igual $0$ se $z_0 \notin \G$ e é igual ao $f(z_0)$ se $z_0 \in G \setminus \partial G$.
\end{teorema}
\begin{proof}
Se $z_0$ é fora de $G$ então é o corolário do \Cref{t:cauchy-k}.
Se $z_0$ é dentro, consideremos uma bola fechada $\overline{B}(z_0,\rho)\subset G \setminus \partial G$.
A diferença entre os cíclos $\partial \overline{B}(z_0,\rho)$ e $\partial G$ é uma fronteira,
e.g. pode realizar a diferença como fronteira de compacto $K$ tais que $\frac{f(z)}{z-z_0}$ é holomorfa em $K$,
e $\partial G - \partial \overline{B}(z_0,\rho) = \partial K$.
Logo 
\[ 2\pi i f(z_0) = \int_{\partial \overline{B}(z_0,\rho)} \frac{f(z)dz}{z-z_0} = \int_{\partial G} \frac{f(z)dz}{z-z_0}. \]
\end{proof}

\section{Decomposição em séries de Taylor.}

\section{Teorema de Morera. Critério de holomorficidade.}

\section{Teorema de Weierstrass.}

\section{Funções holomorfas em um anel. Séries de Laurent.}
% setembro 13

\begin{teorema}
Seja $f$ uma função holomorfa num anel $A = \{ 0\leq r < |z-a| < R \leq \infty \}$. Então em $A$ temos
\[ f(z) = \sum_{n=-\infty}^\infty c_n (z-a)^n, \]
onde
\[ c_n = \frac1{2\pi i} \int_{\gamma_\rho} \frac{f(w) dw}{(w-a)^{n+1}} \]
e
\[ \gamma_\rho = \{ z\in\C \,|\, |z-a|=\rho \} \subset A. \]
\end{teorema}
\begin{proof}
Para $z\in A$ escolha um subanel fechado $A' = \{ r < r'\leq |w-a| \leq R' < R \}$.
Pela fórmula de Cauchy temos

\end{proof}


\section{Pontos singulares isolados.}

\section{Resíduos. Valor principal de Cauchy.}

\section{Princípio do argumento.}
\begin{defin}
\end{defin}
\begin{teorema}
\end{teorema}
\begin{teorema}[Rouché]
\end{teorema}
\begin{cor}[Teorema Fundamental de Álgebra]
\end{cor}

\section{Propriedades topológicas das funções meromorfas}

\section{Teorema de Montel. Funcionais contínuos nas famílias compactas de funções.}

\section{Teorema de Hurwitz. Funções univalentes.}

\section{Continuação analítica.}

\section{Teorema de Riemann.}

\section{Automorfismos de domínios simplesmente conexos.}

\section{Superfícies de Riemann. Uniformização.}

\begin{teorema}[Poincaré, Koebe 1907, foi 22o problema de Hilbert]
Qualquer superfície de Riemann simplesmente conexa é biholomorfa ao uma de três superfícies padrão:
esfera de Riemann $\C\P^1 = \ol{C}$, reta/plano complexo $\C$ ou disco $D = \{z : |z|<1\}$.
\end{teorema}
Os métodos de demonstração deste teorema são fora deste curso - temos que construir alguns funções
harmônicas, e para isso usamos EDP. Veja e.g. livro texto de Donaldson "Riemann Surfaces" (2011).

\section{Grupos de Fuchs.}

\begin{exem}[Grupo modular]
$\Gamma = \{ z \mapsto \frac{az+b}{cz+d} \,|\, a,b,c,d\in\Z, \, ad-bc>0 \} = PSL(2,\Z)$
\end{exem}

\section{Espaço de módulos de toros complexos.}

\begin{teorema}
O espaço de módulos de toros complexos é identificado com o quociente $\H/\Gamma$
do plano superior por grupo modular.
\end{teorema}

\section{Funções analíticas.}

\section{Exemplos das superfícies de Riemann das funções analíticas.}
