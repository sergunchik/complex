\section{Fórmula de Euler para cotangente e seus corolários}
% novembro 17

A seguinte discussão em particular resolva \Cref{p11}.


As funções
\begin{equation*}
g_n(z) := \sum_{k=-n}^n (z+k)^{-1} 
\end{equation*}
são holomorfas em $\C - \Z$ e $g_n$ tem polos simples com resíduo $1$
em $k=-n,\dots,n$.

Podemos combinar os termos
\[ \frac{1}{z-k} + \frac{1}{z+k} = \frac{-2z}{k^2-z^2} \]
para ver que para qualquer compacto $K \subset \C-\Z$ com $M := \max_{z\in K} |z|$ temos
para $n\geq M$
\[ |g_{n+m}(z) - g_n(z)| \leq 2|z| \sum_{k=n}^\infty (k-|z|)^{-2}
\leq 2 M \int_{n-1-M}^\infty x^{-2} dx = \frac{2 M}{n-1-M} = O(1/n), \]

logo pelo critérios de Cauchy e Weierstrass a série
$g_n(z)$ converge absolutamente e localmente uniformemente a uma função
\[ g(z) := \lim_{n\to\infty} g_n(z) .\]

A gente quer provar que $g(z) = \pi \cot(\pi z)$,
e a nossa estrategia será mostrar que $g(z)$ e $h(z) = \pi \cot(\pi z)$
satisfazem suficiente propriedades comuns para derivar que a diferença delas é zero.

Primeiro vamos provar que $g$ é periodica: $g(z+1) = g(z)$.
Temos
\begin{multline*}
g(z+1) - g(z) = \lim_{n\to\infty} g_n(z+1) - g_n(z) \\
= \lim_{n\to \infty} \big((z+1-n)^{-1} + (z+2-n)^{-1} + \dots + (z+n)^{-1} + (z+n+1)^{-1}\big)
\\ -\big((z-n)^{-1} + (z+1-n)^{-1} + \dots + (z+n-1)^{-1} + (z+n)^{-1}\big)
\\  = (z+n+1)^{-1} - (z-n)^{-1} = \frac{-2n-1}{(z+n+1)(z-n)} = 2/n \cdot \frac{1+1/(2n)}{1+1/n+(z^2+z)/n^2} = O(1/n)
\end{multline*}
no qualquer compacto $z\in K \subset \C-\Z$.
Similarmente
\begin{equation}
\cot(\pi z) = i \frac{\exp(2\pi i z)+1}{\exp(2\pi i z)-1}
\end{equation}
e $\exp(2\pi i(z+1) = \exp(2\pi i) \exp(2\pi i z) = \exp(2 \pi i z)$.

Da mesma equação vejamos que $\pi \cot(\pi z)$ 
é holomorfa em $\C-\Z$,
tem polo simples em $n\in\Z$ com resíduos igual a $1$.

As cotas que usamos para provar convergência de $g_n(z)$ para $g(z)$ também mostram
que $\tilde{g}_n(z) = g_n(z) - \frac{1}{z}$ converge a função $\tilde{g}(z) = g(z)-\frac{1}{z}$
holomorfa também em $z=0$, logo $g(z) = \tilde{g}(z) + \frac{1}{z}$ é meromorfa em $z$
com polo simples e resíduo $1$. Desde que $g(z)$ é periodica ela também é meromorfa em outros inteiros
com o mesmo resíduo $1$.

Logo a função definida como a diferença

\[ d(z) := g(z) - h(z) = g(z) - \pi \cot(\pi z) \]

é uma função inteira, desde que as partes principais das funções meromorfas $g(z)$ e $h(z)$ se cancelam em cada ponto.


Similarmente a demonstração de periodicidade podemos provar

\[ 2 g(z) = g(\frac{z}{2}) + g(\frac{z+1}{2}). \]

De fato,
\begin{align*}
g_n(z/2) &= \sum_{k=-n}^n (z/2+k)^{-1} = 2 \sum_{k=-n}^n (z+2k)^{-1}, \\
g_n((z+1)/2) &= \sum_{k=-n}^n ((z+1)/2+k)^{-1} = 2 \sum_{k=-n}^n (z+2k+1)^{-1}, \\
g_n(z/2) + g_n((z+1)/2) &= 2 \sum_{l=-2n}^{2n+1} (z+l)^{-1}, \\
g_n(z/2) + g_n((z+1)/2) - 2 g_{2n}(z) &= 2 (z+2n+1)^{-1}  = O(1/n)
\end{align*}
no qualquer compacto.

Denotando $q := \exp(\pi i z)$, temos
\begin{align*}
-i \cot(\pi z/2) &= \frac{\exp(\pi i z)+1}{\exp(\pi i z)-1} = \frac{q+1}{q-1}, \\
-i \cot(\pi (z+1)/2) &= \frac{\exp(\pi i (z+1))+1}{\exp(\pi i (z+1))-1} = \frac{-q+1}{-q-1}, \\
-i \cot(\pi z) &= \frac{\exp(2 \pi i z)+1}{\exp(2 \pi i z)-1} = \frac{2(q^2+1)}{q^2-1},
\end{align*}
logo
\[ (\cot(\pi z/2) + \cot(\pi (z+1)/2)
   = i \frac{q+1}{q-1} + i \frac{q-1}{q+1}
   = i \frac{(q+1)^2+(q-1)^2}{(q-1)(q+1)}
   = i 2 \frac{q^2+1}{q^2-1}
   = 2 \cot(\pi z), \]

Então a função $d(z) = g(z) - h(z)$ é
\begin{enumerate}
\item inteira $d(z) \in \O(\C)$,
\item periodica $d(z+1) = d(z)$,
\item satisfaz $d(\R) \subset \R$.
\end{enumerate}

Isso implica que existe o valor máximo $C := \max_{z\in [0,1]} d(z) = \max_{z\in\R} d(z)$
e ele é atingido em algum ponto real $w$.
Sem perda de generalidade podemos supor que $w\neq 0$, se $w=0$ usamos periodicidade $d(1) = d(0)$.
A equação
\[ d(w/2) + d((w+1)/2) = 2 d(w) \]
e as desigualdades $d(w/2) \leq C = d(w)$ e $d((w+1)/2) \leq C = d(w)$ implicam
que $d(w/2) = C$. Similarmente $d(w/4) = C$, etc. Então sabemos que
$d(w/2^k)=0$. Pelo teorema de identidade obtemos que $d(w)$ é igual a constante $C$.
Desde que $g(z)$ e $h(z)$, e logo $d(z)$, são funções impares temos $C = d(0) = 0$,
isso é $g(z) = h(z)$:

\begin{equation}
\sum_{k\in\Z} (z+k)^{-1} = \pi \cot(\pi z),
\end{equation}
onde no lado direito a soma converge em sentido de Eisenstein
(i.e. igual ao $\lim \sum_{k=-n}^n$).

Logo $g(1/4) = \pi \cot(\pi/4) = \pi$.
No outro lado
$4 g(1/4) = 1 - 1/3 + 1/5 - \dots$,
em particular obtemos a soma de Madhava--Gregory--Leibniz em outro jeito
\[ 1 - 1/3 + 1/5 - 1/7 + \dots = \frac{\pi}{4}. \]

O jeito de Madhava foi escrever a série de Taylor de função arctangente
(cf. os problemas em \Cref{ss:elementar})
\[ \tan^{-1}(x) = x - \frac{x^3}3 + \frac{x^5}5 - \dots, \]
e obter
\[ 1 - 1/3 + 1/5 - 1/7 + \dots = \tan^{-1}(1) = \pi/4. \]


Vamos considerar a série de Laurent da função $g(z)$. Temos
\begin{equation}
z \cdot g(z) = 1 - 2 \sum_{n=1}^\infty \frac{z^2}{n^2-z^2} = 1 - 2 \sum_{n,k=1^\infty} n^{-2k} z^{2k}
= 1 - 2 \sum_{k=1}^\infty \zeta(2k) z^{2k}
\end{equation}
onde
\begin{equation}
\zeta(s) := \sum_{n=1}^\infty n^{-s}
\end{equation}
é a função zeta de Riemann.

Logo
\begin{equation}
w \cot(w) = 1 - 2 \sum_{k=1}^\infty \frac{\zeta(2k)}{\pi^{2k}} z^{2k}.
\end{equation}
Mas no outro lado a série de Taylor
\[ w \cot(w) = 1 - \frac{1}{3} w^2 - \frac{1}{45} w^4 - \frac{2}{945} w^6 
  -\frac{1}{4725} w^8 - \frac{2}{93555} w^{10} - \frac{1382}{638512875} w^{12} 
 -\dots \]
tem os coeficientes racionais desde
que $\cos(w) = 1 - w^2/2 + \dots = \sum \frac{(-1)^n}{(2n)!} w^{2n}$ e
$\sin(w) = w - w^3/6 + \dots = \sum \frac{(-1)^n}{(2n+1)!} w^{2n+1}$
tem os coeficientes racionais.
Então provamos que para qualquer $k\geq 1$ a razão $\frac{\zeta(2k)}{\pi^{2k}}$
é um número racional, e este número essencialmente é o coeficiente de Laurent--Taylor--Maclaurin
da função cotangente:
\[
2 \frac{\zeta(2)}{\pi^2} = \frac{1}{2},
2 \frac{\zeta(4)}{\pi^4} = \frac{1}{45},
2 \frac{\zeta(6)}{\pi^6} = \frac{2}{945},
2 \frac{\zeta(8)}{\pi^8} = \frac{1}{4725},
2 \frac{\zeta(10)}{\pi^{10}} = \frac{2}{93555},
2 \frac{\zeta(12)}{\pi^{12}} = \frac{1382}{638512875}, 
\dots \]

Formalmente os \emph{números de Bernoulli} são definidos pelo
\begin{align}
\frac{z}{1-\exp(-z)} = z/2 \cdot ( \coth(z/2)+1 ) &=: \sum_{m=0}^\infty \frac{B_m^+}{m!} \cdot z^m, \\
\frac{z}{\exp(z)-1} = z/2 \cdot ( \coth(z/2)-1 ) &=: \sum_{m=0}^\infty \frac{B_m^-}{m!} \cdot z^m. \\
\end{align}
Temos $B_1^+ = 1/2$, $B_1^- = -1/2$ e $B_m^+ = B_m^-$ para $m\neq 1$,
em aplicações matemáticas $B_m^+$ é mais natural,
mas as funções $\frac{z}{1-\exp(-z)}$ e $\frac{z}{\exp(z)-1}$ obtidas uma de outra
pela suibstituição $z\mapsto -z$.
Com essa normalização temos
\begin{equation}
B_{2n} = \frac{(-1)^{n+1} 2 (2n)!}{(2\pi)^{2n}} \zeta(2n).
\end{equation}

Em particular obtemos as fórmulas de Euler\footnote{
O problema de computar a soma
\[ \zeta(2) = \frac{1}{1^2} + \frac{1}{2^2} + \frac{1}{3^2} + \dots \]
é conhecido como o \emph{problema de Baseleia},
a cidade natal de Euler e família Bernoulli.
Ele foi proposta por Pietro Mengoli em 1650 e resolvida
por Leonhard Euler em 1734, sete anos desde que chegou a trabalhar
em São Petersburgo, e é publicado em artigo ``De summis serierum reciprocarum''
em 1734, conhecido como E41 (41o artigo de Euler).
Série infinita para cotangente e produto infinito para seno
também foram descobertas por Euler.

O primeiro código fonte para computador é o programa para computar os números
de Bernoulli escrito por Ada Lovelace em 1842 para a máquina de Babbage.}

\[ \zeta(2)  = \frac{\pi^2}{6},
   \zeta(4)  = \frac{\pi^4}{90},
   \zeta(6)  = \frac{\pi^6}{945},
   \zeta(8)  = \frac{\pi^8}{9450},
   \zeta(10) = \frac{\pi^{10}}{93555},
   \zeta(12) = \frac{691 \cdot \pi^{12}}{638512875},
\dots \]


\begin{problema}[Séries de Eisenstein]
Para inteiro $k\geq 2$ e $\Im(z)\neq 0$ define as \emph{séries de Eisenstein} pelo
\begin{equation}
E_k(z) := \sum_{c,d\in \Z, (c,d)\neq(0,0)} (cz+d)^{-k} .
\end{equation}
Provar que
\begin{enumerate}
\item a série $E_k(z)$ converge
absolutamente e localmente uniformemente no semiplano superior
$\H = \{z\in\C: \Im(z)>0\}$ para qualquer inteiro $k>2$;
\item a série $E_2(z)$ converge em sentido de Eisenstein,
i.e. existe o limite $E_2(z) = \lim_{n\to\infty} \sum_{c=-n}^n\sum_{d=-n}^n (cz+d)^{-2}$
onde a soma não tem termo $(c,d)=(0,0)$;
\item $E_{2l+1}(z) = 0$;
\item a soma $E_k(z)$ é uma função holomorfa $E_k \in \O(\H)$;
\item as funções $E_k(z)$ são periódicas, i.e. $E_k(z+1) = E_k(z)$;
\item existe uma função holomorfa no disco unitário perfurado
$G_k(q) \in \D^* := \{q\in\C: 0 < |q| < 1 \}$ tal que $E_k(z) = G_k(\exp(2\pi i z))$;
\item $G_k(q)$ tem singularidade removível em $q=0$.
\end{enumerate}
\end{problema}

\begin{problema}
Calcular as séries de Taylor das funções $G_k(z)$.
\end{problema}


