\documentclass[brazilian,a4paper]{scrlttr2}
\usepackage{babel}
\usepackage{microtype}
\pagestyle{empty}
\newcounter{problema}
\newenvironment{problema}[1][]
{\refstepcounter{problema}
\par\textbf{\theproblema. #1}\rmfamily}{}

\def\R{\mathbf{R}}
\def\C{\mathbf{C}}
\def\D{\mathbf{D}}
\def\H{\mathbf{H}}
\def\Z{\mathbf{Z}}
\def\Ch{\widehat{\C}}
\def\O{\mathcal{O}}
\begin{document}
\emph{Notações.}
As regiões comuns:
$\Ch = \C \bigcup \infty$ --- a \emph{esfera de Riemann},
$H = \{z\in\C: \Im(z)>0\}$ --- o \emph{plano superior},
$D_R = \{z\in\C: |z|<R\}$ --- o \emph{disco} de raio $R>0$,
$\D$ --- o \emph{disco unitário},
$\D^* = \D - \{0\}$ --- o \emph{disco perfurado},
$A_R := \{z\in\C: 1/R < |z| < R \}$ o \emph{anel} para $R\in(1,+\infty)$,
$D\subset \Ch$ --- uma região;
$\O(D)$ --- o conjunto das funções holomorfas na $D$;
$\phi : \Ch\to\Ch$ --- as \emph{transformações de Möbius} 
$\phi(z) = \frac{az+b}{cz+d}$
com $a,b,c,d\in\C$ e $ad\neq bc$.

\bigskip
\emph{Os problemas}.

\begin{problema}%[1]
Para $x\in(-1,1)$
construir um biholomorfismo %elementar entre
$\psi: \H \to \D - [x,1)$.
% o plano superior
% e o complemento para o intervalo $[x,1) \subset \R$
% no disco unitário.
\end{problema}

\begin{problema}%[2]
Para círculo generalizado (círculo ou reta) $S \subset \Ch$ define
uma \emph{refleção}
$R_S : \Ch \to \Ch$ como a refleção em reta ou a inversão em círculo.
% Seja $\phi: \Ch \to \Ch$ uma transformação de Möbius, i.e. $\phi(z) = \frac{az+b}{cz+d}$
%com $ad\neq bc$.
Provar que as transformações de Möbius $\phi : \Ch\to\Ch$
preservam as refleções, i.e. para qualquer
círculo $S$ e imágem dela $\phi(S)$ temos $\phi(R_S z) = R_{\phi(S)} \phi(z)$.
\end{problema}

\begin{problema}%[3]
Sejam $C_1,C_2 \subset \Ch$ um par de círculos disjuntos, i.e. $C_1 \bigcap C_2 = \emptyset$,
e $D \subset \Ch$ uma região com fronteirs $\partial D = C_2 - C_1$.
(a) Provar que existe uma transformação de Möbius $\phi$ tal que
$\phi(D) = A_R$ para algum $R>1$. %  = \{ z\in\C: 1/R < |z| < R \}$.
(b) Qual é o lugar geométrico de círculos $S$, que tocam ambos $C_1$ e $C_2$?
\end{problema}

\begin{problema}%[4]
Define $Sf = \frac{f'''}{f'} - \frac{3}{2} \big(\frac{f''}{f'}\big)^2$.
Provar para qualquer funções $f,g$ com singularidades isoladas e qualquer
transformação de Möbius $\phi$ que
(a) $S \phi = 0$,
(b) $S (\phi \circ f) = S f$,
(c) $S(f \circ \phi) = \big((Sf)\circ\phi\big) \cdot (\phi')^2$,
(d) $S(f\circ g) = \big((Sf)\circ g\big) \cdot (g')^2 + Sg$.
\end{problema}

\begin{problema}%[5]
Achar $\lim_{\rho\to+0} \int_\alpha^\beta f(\rho\cdot e^{i\phi}) d\phi$
para $f : D_\epsilon\to\C$ contínua.
\end{problema}

\begin{problema}%[6]
Seja $\omega_t = \frac{e^{itz} dz}{(z+z^{-1})^3}$.
(a) Classificar singularidades de $\omega_t$ na $\Ch$
para cada $t\in\C$.
(b) Calcular $I(t) = \int_\R \omega_t$.
\end{problema}

\begin{problema}%[7]
Para $f\in\O(\D^*)$ % uma função holomorfa no disco perfurado $D^*$. % = P(0,1) = \{z\in\C: 0<|z|<1\}$.
suponha que $\frac{f'(z)}{f(z)}$
tem um polo simples em $0$. Provar que
(a) a função $f$ é meromorfa em $0$,
(b) e $f$ tem zero ou polo em $0$.
\end{problema}

\begin{problema}%[8]
Seja $f(z) = \frac{P(z)}{Q(z)}$ para um par de polinómios $P,Q\in\C[z]-\{0\}$.
Suponha que para qualquer $|z|=1$ temos $|f(z)|=1$. Classificar $f$,
i.e. dar uma condição suficiente e necessária sobre os polos e nulos de $f$.
\end{problema}

\begin{problema}%[9]
%Quais podem ser as funções inteiras 
Classificar $f,g\in\O(\C)$ tais que
$f^2+g^2=1$ e $f(z+w) = f(z) g(w) + f(w) g(z)$.
\end{problema}

\begin{problema}%[10]
Suponha que $f \in \O(A_R)$ %é holomorfa no anel $A_R$,
% $A = \{z\in\C: 1 < |z| < R\}$ com $r<1<R$,
tem a série de Laurent $f(z) = \sum c_n z^n$ para $z\in A_R$.
Define $F(z) = \sum_{n\geq 0} c_n z^n$ e $H(z) = \sum_{n>0} c_{-n} z^{-n}$ e $G(z) = H(z^{-1}) = \sum_{n>0} c_{-n} z^{n}$.
(o) Verifique que $F,G,H$ são holomorfas em $A_R$.
(a) Define $g(z) := \frac{1}{2\pi i} \int_{|w|=1} \frac{f(w) dw}{w-z}$
para $z\in A - \{|z|=1\}$. Provar que $g(z) = F(z)$ para $|z|<1$ e $g(z)=H(z)$ para $|z|>1$.
(b) Provar que para qualquer $|z|=1$ temos $f(z) = F(z) + H(\bar{z})$.
\end{problema}

\begin{problema}%[11]
Para $z\in\C-\Z$
(a) a série $\sum_{n\in\Z} \frac{1}{z+n}$ converge absolutamente?
(b) existe $f(z) = \lim_{n\to\infty} S_n$ onde $S_n = \sum_{k=-n}^n \frac{1}{z+n}$?
(c) é localmente uniforme em $\C-\Z$?
(e) a função $f$ tem polos em $\C$?
(f) a função $f$ tem zeros?
(g) $f(z+1) = f(z)$?
(d) qual é o valor $f(1/4)$?
\end{problema}
\end{document}
