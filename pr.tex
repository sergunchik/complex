\documentclass[brazilian,a4paper]{scrlttr2}
% \usepackage{babel,microtype,enumitem,amsmath,amsthm,cleveref}
\pagestyle{empty}
\usepackage{import}
\import{p}{p}
% % usar \defcommand em vez de \newcommand ou \renewcommand
% funciona como \def em tex - vai definir, se ainda não definida,
% se já definida, vai redefinir
\makeatletter\def\defcommand{\@ifstar\defcommand@S\defcommand@N} \def\defcommand@S#1{\let#1\outer\renewcommand*#1} \def\defcommand@N#1{\let#1\outer\renewcommand#1} \makeatother

\mathchardef\mhyphen="2D
\newcommand\dash{\nobreakdash-\hspace{0pt}}

\providecommand{\tightlist}{%
  \setlength{\itemsep}{0pt}\setlength{\parskip}{0pt}}

% mathcal e mathb letras
\def \CC {\mathcal{C}}  % 

%mathbf letras

\def \A {\mathbf{A}}    % anel
\def \C {\mathbf{C}}    % números complexos
\def \Ch {\widehat{\C}} % esfera de Riemann
\def \D {\mathbf{D}}    % disco
\def \F {\mathbf{F}}    % corpos finitos F_q
\def \H {\mathbf{H}}	% semiplano superior
\def \P {\mathbf{P}}    % números primos
\def \Q {\mathbf{Q}}    % números rationais
\def \R {\mathbf{R}}    % números reais
\def \Z {\mathbf{Z}}    % números inteiros

\def \M  {\mathcal{M}}    % meromorphic functions
\def \O  {\mathcal{O}}    % holomorphic functions

\newcommand\cat{\iftutex\DejaSans\text{😺}\else\mathcal{G}\fi}
\newcommand\gato{\iftutex\DejaSans{😺}\else{G}\fi}


\newcommand{\lra}{\longrightarrow}
\newcommand\ii{\ensuremath{\mathrm{i}}}  % raiz quadrática de -1
\newcommand\dpi{2\pi\ii}                 % 2πi = ∫ dz/z
\newcommand\idpi{\frac{1}{\dpi}}         % 1/(2πi)
\newcommand\ol{\overline}

\DeclareMathOperator{\dlog}{dlog}
\DeclareMathOperator{\res}{res}
\DeclareMathOperator{\sinal}{sinal}
\renewcommand{\ord}{ord}
\DeclareMathOperator{\mdc}{mdc}
\DeclareMathOperator{\mmc}{mmc}
\DeclareMathOperator{\codim}{codim}
\DeclareMathOperator{\Hom}{Hom}
\DeclareMathOperator{\End}{End}
%\defcommand{\Im}{\mathrm{Im} }
\let\Im\relax
\DeclareMathOperator{\Im}{Im}
\DeclareMathOperator{\Ker}{Núc}
\DeclareMathOperator{\Coker}{Conúc}
\DeclareMathOperator{\Aut}{Aut}
\DeclareMathOperator{\Tr}{Tr}
\DeclareMathOperator{\tr}{tr}
%\DeclareMathOperator{\dim}{dim}
\DeclareMathOperator{\PGL}{PGL}
\DeclareMathOperator{\GL}{GL}

\defcommand{\mat}[1]{\big(\begin{smallmatrix} #1 \end{smallmatrix}\big)}

\providecommand{\arxiv}[1]{\href{http://arxiv.org/abs/#1}{arXiv:#1}}
\providecommand{\doi}[1]{\href{http://dx.doi.org/#1}{\texttt{doi:#1}}}



\begin{document}
\emph{Notações.}
As regiões comuns:
$\Ch = \C \bigcup \infty$ --- a \emph{esfera de Riemann},
$H = \{z\in\C: \Im(z)>0\}$ --- o \emph{plano superior},
$D_R = \{z\in\C: |z|<R\}$ --- o \emph{disco} de raio $R>0$,
$\D$ --- o \emph{disco unitário},
$\D^* = \D - \{0\}$ --- o \emph{disco perfurado},
$\A_R := \{z\in\C: 1/R < |z| < R \}$ o \emph{anel} para $R\in(1,+\infty)$,
$D\subset \Ch$ --- uma região;
$\O(D)$ --- o conjunto das funções holomorfas na $D$;
$\phi : \Ch\to\Ch$ --- as \emph{transformações de Möbius} 
$\phi(z) = \frac{az+b}{cz+d}$
com $a,b,c,d\in\C$ e $ad\neq bc$.

\bigskip
\emph{Os problemas} de outubro 18.

\begin{prob}\label{p1} % [1]
Para $x\in(-1,1)$
construir um biholomorfismo %elementar entre
$\psi: \H \to \D - [x,1)$.
% o plano superior
% e o complemento para o intervalo $[x,1) \subset \R$
% no disco unitário.
\end{prob}

\begin{prob}\label{p2} % [2]
Para círculo generalizado (círculo ou reta) $S \subset \Ch$ define
uma \emph{refleção}
$R_S : \Ch \to \Ch$ como a refleção em reta ou a inversão em círculo.
% Seja $\phi: \Ch \to \Ch$ uma transformação de Möbius, i.e. $\phi(z) = \frac{az+b}{cz+d}$
%com $ad\neq bc$.
Provar que as transformações de Möbius $\phi : \Ch\to\Ch$
preservam as refleções, i.e. para qualquer
círculo $S$ e a imágem dela $\phi(S)$ temos $\phi(R_S z) = R_{\phi(S)} \phi(z)$.
\end{prob}

\begin{prob}\label{p3} % [3]
Sejam $C_1,C_2 \subset \Ch$ um par de círculos disjuntos, i.e. $C_1 \bigcap C_2 = \emptyset$,
e $D \subset \Ch$ uma região com a fronteira $\partial D = C_2 - C_1$.
(a) Provar que existe uma transformação de Möbius $\phi$ tal que
$\phi(D) = \A_R$ para algum $R>1$. %  = \{ z\in\C: 1/R < |z| < R \}$.
(b) Qual é o lugar geométrico de círculos $S$, que tocam ambos $C_1$ e $C_2$?
\end{prob}

\begin{prob}\label{p4} % [4]
Define $Sf = \frac{f'''}{f'} - \frac{3}{2} \big(\frac{f''}{f'}\big)^2$.
Provar para qualquer funções $f,g$ com singularidades isoladas e qualquer
transformação de Möbius $\phi$ que
(a) $S \phi = 0$,
(b) $S (\phi \circ f) = S f$,
(c) $S(f \circ \phi) = \big((Sf)\circ\phi\big) \cdot (\phi')^2$,
(d) $S(f\circ g) = \big((Sf)\circ g\big) \cdot (g')^2 + Sg$.
\end{prob}

\begin{prob}\label{p5} % [5]
Achar $\lim_{\rho\to+0} \int_\alpha^\beta f(\rho\cdot e^{i\phi}) d\phi$
para $f : D_\epsilon\to\C$ contínua.
\end{prob}

\begin{prob}\label{p6} % [6]
Seja $\omega_t = \frac{e^{itz} dz}{(z+z^{-1})^3}$.
(a) Classificar singularidades de $\omega_t$ na $\Ch$
para cada $t\in\C$.
(b) Calcular $I(t) = \int_\R \omega_t$.
\end{prob}

\begin{prob}\label{p7} % [7]
Para $f\in\O(\D^*)$ % uma função holomorfa no disco perfurado $D^*$. % = P(0,1) = \{z\in\C: 0<|z|<1\}$.
suponha que $\frac{f'(z)}{f(z)}$
tem um polo simples em $0$. Provar que
(a) a função $f$ é meromorfa em $0$,
(b) e $f$ tem zero ou polo em $0$.
\end{prob}

\begin{prob}\label{p8} % [8]
Seja $f(z) = \frac{P(z)}{Q(z)}$ para um par de polinómios $P,Q\in\C[z]-\{0\}$.
Suponha que para qualquer $|z|=1$ temos $|f(z)|=1$. Classificar $f$,
i.e. dar uma condição suficiente e necessária sobre os polos e nulos de $f$.
\end{prob}

\begin{prob}\label{p9} % [9]
%Quais podem ser as funções inteiras 
Classificar $f,g\in\O(\C)$ tais que
$f^2+g^2=1$ e $f(z+w) = f(z) g(w) + f(w) g(z)$.
\end{prob}

\begin{prob}\label{p10} % [10]
Suponha que $f \in \O(\A_R)$ %é holomorfa no anel $\A_R$,
% $A = \{z\in\C: 1 < |z| < R\}$ com $r<1<R$,
tem a série de Laurent $f(z) = \sum c_n z^n$ para $z\in \A_R$.
Define $F(z) = \sum_{n\geq 0} c_n z^n$ e $H(z) = \sum_{n>0} c_{-n} z^{-n}$ e $G(z) = H(z^{-1}) = \sum_{n>0} c_{-n} z^{n}$.
(o) Verifique que $F,G,H$ são holomorfas em $\A_R$.
(a) Define $g(z) := \frac{1}{2\pi i} \int_{|w|=1} \frac{f(w) dw}{w-z}$
para $z\in A - \{|z|=1\}$. Provar que $g(z) = F(z)$ para $|z|<1$ e $g(z)=H(z)$ para $|z|>1$.
(b) Provar que para qualquer $|z|=1$ temos $f(z) = F(z) + H(\bar{z})$.
\end{prob}

\begin{prob}\label{p11} % [11]
Para $z\in\C-\Z$
(a) a série $\sum_{n\in\Z} \frac{1}{z+n}$ converge absolutamente?
(b) existe $f(z) = \lim_{n\to\infty} S_n$ onde $S_n = \sum_{k=-n}^n \frac{1}{z+n}$?
(c) é localmente uniforme em $\C-\Z$?
(e) a função $f$ tem polos em $\C$?
(f) a função $f$ tem zeros?
(g) $f(z+1) = f(z)$?
(d) qual é o valor $f(1/4)$?
\end{prob}

\emph{Os probs bônus}.

\begin{prob} ($i^i=?$)
Sejam $f,g,h \in \O(\C-(-\infty,0])$ tais que para $u\in\R$ e $x=e^u$
temos $f(x) = x^x$, $g(x) = e^{ui}$, $h(x) = e^{\frac{\pi i}{2} x}$.
Calcular $f(i)$, $g(i)$ e $h(i)$.
\end{prob}

\begin{prob}
Mostre que $f(z) = \frac{z}{1-\exp(-z)}$ é uma função inteira
e calcule $f^{(101)} (0)$ (cento-primeira derivada em zero).
\end{prob}

%\newpage
\bigskip

\emph{Os tipos de probs}:
\begin{enumerate}
\item Determinar tipos de pontos singulares e calcular os resíduos.
\item Calcular a integral.
\item Construir biholomorfismo entre duas regiões.
\item Calcular a soma de série.
\item Decompor uma função em série de Laurent nos anéis com dado centro.
\item Descrever a superfície de Riemann de dada função analítica.
\end{enumerate}

\setcounter{prob}{0}

\medskip

\emph{Os exemplos de problemas finais} (com uma estimação do nível de dificuldade)

\begin{prob} (17)
Sejam $u_0,\dots,u_n\in\C$ distintas,
$P = \prod_{k=0}^n (z-u_k)$ ,
$f(z)$ uma função, $Q \in \C[z]$ um polinómio de grau $n$ tal que
$\forall k: Q(u_k) = f(u_k)$.
Provar que
$f(z) - Q(z) = \frac{1}{2\pi i} P(z) \int_{|w|=R} \frac{f(w) dw}{P(w) \cdot (w-z)}$
para qualquer $R > \max(u_0,\dots,u_n,z)$.
\end{prob}

\begin{prob} (20)
Provar que todas as raízes complexas de $\tan(z) = z$ são reais.
\end{prob}

\begin{prob} (7)
Será que existe uma região $\R-\{0\} \subset D\subset \C$ e $f\in\O(D)$ tal que
$f(x) = \log (x^2) \in \R$ para cada $x\in\R-\{0\}$?
\end{prob}

\begin{prob}[(9)(mestrado)]
Provar que o cobrimento universal de um toro analítico com um ponto perfurado
é $\D$.
\end{prob}

\begin{prob} (11)
Provar que 
$f,g\in\O(\C)$ t.q. $\exp(f(z)) + \exp(g(z)) = 1$
são constantes.
\end{prob}

\begin{prob} (13)
Provar que existe uma função $f$ meromorfa no $\C$ t.q. $f : \Delta \to \D$ é uma bijeção,
onde $\Delta$ é um triângulo com ângulos $\frac{\pi}{6},\frac{\pi}{3},\frac{\pi}{2}$.
\end{prob}

\begin{prob} (5)
Caraterizar os pontos singulares de $f(z) = \exp(\tan(1/z))$.
\end{prob}

\begin{prob} (6)
Achar o número de raízes do polinómio $z^3-z^2+3z+5$ com $\Re(z)>0$.
\end{prob}

\begin{prob} (3)
Provar que $f\in\O(\C)$ t.q. $|f(z)|\leq C |z|^k$ é um polinómio.
\end{prob}

\begin{prob} (4)
Achar todas as séries de Laurent nos anéis com centro em zero da função $f(z) = \frac{7z-2}{(z+1)(z-2)}$.
\end{prob}

\begin{prob} (5)
Provar que $u(x,y) = \Re f(x+iy)$ para $f\in\O(\D)$ $\iff$ 
$\frac{\partial^2 u}{\partial x^2} + \frac{\partial^2 u}{\partial y^2} = 0$.
\end{prob}

\subsection{Sergey}
\pnove*
\begin{proof}[\Cref{p9}]

Uma solução óbvia é $f(z) = \sin(z)$ e $g(z) =\cos(z)$.
Também as equações tem uma simetria no domínio:
se $f(z),g(z)$ é uma solução, então $f_k(z) = f(kz), g_k(z) = g(kz)$
também é uma solução para qualquer $k\in\C$.
Fórmula trigonométrica do seno da soma
$\sin(z+w) = \sin(z) \cos(w) + \cos(z) \sin(w)$
tem uma versão para cosseno:
\[ \cos(z+w) = \cos(z)\cos(w) - \sin(z)\sin(w). \]
A priori a gente não sabe $g(z+w)$, mas podemos derivar o valor dele
da equação $g(z+w)^2 = 1 - f(z+w)^2$ substituindo $f(z+w)$
por $f(z)g(w)+g(z)f(w)$:
\begin{multline*}
g(z+w)^2 = 1\cdot 1 - f(z+w)^2 = (f(z)^2+g(z)^2)(f(w)^2+g(w)^2) - f(z+w)^2
\\ = \big(f(z)^2+g(z)^2\big)\cdot\big(f(w)^2+g(w)^2\big) - \big(f(z)g(w)+g(z)f(w)\big)^2
   = \big(g(z)g(w)-f(z)f(w)\big)^2,
\end{multline*}
então para todos $z,w\in\C$ temos
\[ (g(z+w)-g(z)g(w)+f(z)f(w)) \cdot (g(z+w) + g(z)g(w)-f(z)f(w)) = 0. \]
I.e. para todos $z,w\in\C$ ou
\[ V(z,w) := g(z+w) - \big(g(z)g(w) - f(z)f(w)\big) = 0 \]
ou 
\[ F(z,w) := g(z+w) + \big(g(z)g(w)-f(z)f(w))\big) = 0. \]
Queremos ver que é $V(z,w)=0$.

\end{proof}

\subsection{Arthur}
\def\e{e}

\begin{proof}[\Cref{p5}]
Fixe $\epsilon>0$.
Como $f$ é contínua em $0$, existe $\delta>0$ tal que $(\lvert z-0 \rvert < \delta)\rightarrow(\lvert f(z)-f(0)\rvert<\epsilon)$.

Tome $\rho<\delta$.
Assim,
\begin{align*}
\lvert \int_{\alpha}^{\beta}f(\rho\e^{i\phi})d\phi -(\beta-\alpha)f(0)\rvert &= \lvert \int_{\alpha}^{\beta}f(\rho\e^{i\phi})d\phi -\int_{\alpha}^{\beta}f(0)d\phi \rvert  \\
&=\lvert \int_{\alpha}^{\beta}(f(\rho\e^{i\phi})-f(0))d\phi\rvert \\
&\leq \int_{\alpha}^{\beta}\lvert (f(\rho\e^{i\phi})-f(0))\rvert d\phi \\
&=\int_{\alpha}^{\beta}\epsilon d\phi \\
&=(\beta -\alpha)\epsilon \\
\end{align*}

Portanto, fazendo $\epsilon$ tender a $0$, concluimos que o limite desejado é $(\beta - \alpha)f(0)$.




\end{proof}


\begin{proof}[\Cref{p7}]

Como $\frac{f'(z)}{f(z)}$ tem polo simples em $0$, podemos escrever $$\frac{f'(z)}{f(z)}=\frac{\alpha}{z}+g(z),$$ onde $$g(z)=\sum_{n\geq 0}a_nz^n.$$

Veja que $g$ possui uma primitiva $G(z)=\sum_{n\geq 0}a_n\frac{z^{n+1}}{n+1}$ no disco. Assim, podemos definir a função $h(z)=exp(G(z))$, que também é holomorfa (já que é a composição de duas funções holomorfas). Temos que $h(z)=\sum_{n\geq 0}b_nz^n$.

Afirmamos que $\alpha$ deve ser inteiro. De fato, existe $\epsilon>0$ tal que $f(z)\neq 0$ para todo z com $\lvert z\rvert <\epsilon$. Assim, seja $\gamma$ a fronteira do disco de raio $\frac{\epsilon}{2}$ percorrida no sentido anti-horário, isto é, $\gamma(t)=\frac{\epsilon}{2}\cdot e^{i\cdot t}$, com $0\leq t\leq 2\pi$. Temos então que $$\alpha=\frac{1}{2\pi i}\int_{\gamma}\left(\frac{\alpha}{z}+g(z)\right) dz=\frac{1}{2\pi i}\int_{\gamma}\frac{f'(z)}{f(z)}dz=\frac{1}{2\pi i}\int_{f\circ\gamma}\frac{dw}{w}=I(f \circ \gamma, 0)\in Z.$$

Temos $$0=\frac{f'}{f}-\frac{\alpha}{z}-g=\frac{f'}{f}-\frac{(z^{\alpha})'}{(z^{\alpha})}-\frac{h'}{h}=\frac{(\frac{f}{(z^{\alpha}h)})'}{\frac{f}{(z^{\alpha}h)}}$$ donde $(\frac{f}{(z^{\alpha}h)})'=0$, o que significa que $$f=cz^{\alpha}h=\sum_{n \geq 0}(cb_n)z^{n+\alpha}$$ para algum $c\in C-\{0\}$. Isso mostra que $f$ é meromorfa em $0$.

Se $f$ não possui polo em $0$, então $f(z)=\sum_{n\geq 0}c_nz^n$.

Suponha que $f$ não possui raiz em $0$, então $c_0\neq 0$. Assim, $$0=\frac{c_1}{c_0}\cdot 0=lim_{z\rightarrow 0}\left(\frac{f'(z)}{f(z)}z\right)=lim_{z\rightarrow 0}\left(\alpha+zg(z)\right)=\alpha,$$ o que significa que $\frac{f'}{f}$ não teria polo simples, absurdo! Logo, $f$ tem raiz em $0$.

\end{proof}

\subsection{Daniel}
\input{daniel}
\subsection{Henrique}
\input{henrique}
\subsection{Igor}
\input{igor}
\subsection{Joaquim}
\input{joaquim}
\subsection{João Marcelo}
\input{joao}
\subsection{Joel}
\textbf{Definição: }Sejam $U,V\subset\mathbb{C}$ abertos, e $\mathcal{F} = \{f\colon U\longrightarrow V\,:\, f\,\textrm{contínua}\}$. Dizemos que $\mathcal{F}$ é uma família normal de funções se toda sequência $(f_n)_{n\in\mathbb{N}}\in\mathcal{F}$ possui uma subsequência que converge uniformemente em subconjunto compacto de $U$ para uma função contínua $f\colon U\longrightarrow V$.

\begin{proof}[\Cref{p1}]
\textbf{Teorema de Montel: }Seja $U\subset\mathbb{C}$ e $\mathcal{H}(U)$ o conjunto das funções holomorfas definidas em $U$. A família $\mathcal{F}\subset\mathcal{H}(U)$ é normal se, e somente se, $\mathcal{F}$ é localmente uniformemente limitada.

\textbf{Demonstração: } Supondo que $\mathcal{F}$ seja uma família normal de funções, então o Teorema de Arzelà-Ascoli garante que $\mathcal{F}$ é localmente uniformemente limitada.

Reciprocamente, suponha que $\mathcal{F}$ seja localmente uniformemente limitada. O Teorema de Arzelà-Ascoli garante que neste caso, $\mathcal{F}$ é normal. Logo, basta provar que $\mathcal{F}$ é localmente uniformemente equicontínua
\end{proof}

\begin{proof}[\Cref{p2}]

\textbf{Teorema de Riemann:} Seja $U\subset\mathbb{C}$ aberto e simplesmente conexo, tal que $U\neq\mathbb{C}$. Dado $z_0\in U$, existe único biholomorfismo $f\colon U\longrightarrow D$ tal que $f(z_0) = 0$ e $f'(z_0)\in (0,+\infty)$. Onde $D=\{z\in\mathbb{C}\,:\,|z|<1\}$.

\textbf{Demonstração: } Primeiro, fixe $z_0\in U$, e considere a seguinte família de funções holomorfas: $$\mathcal{F}=\{f\in\mathcal{H}(U)\,:\,f(U)\subset D,\,f(z_0)=0,\,f'(z_0)\in(0,+\infty),\,\textrm{f injetiva}\}$$

Mostraremos primeiro a existência. Para isso, vamos mostrar primeiro que $\mathcal{F}\neq\emptyset$.

Como $U\neq\mathbb{C}$, fixe $z_1\in\mathbb{C}\setminus U$. Como $U$ é simplesmente conexo, podemos definir um ramo do logaritmo de $z-z_1$ em U. Assim, considere $L\colon U\longrightarrow \mathbb{C}$ tal ramo. Temos que $\exp({L(z)}) = z-z_1$, $\forall z\in U$.

Afirmação (1): $L$ é injetiva e $\mathbb{C}\setminus L(U)$ tem interior não vazio. 

De fato, dados $z,z'\in U$ tais que $L(z)=L(z')$, segue-se que: $$\exp({L(Z)}) = \exp({L(Z)})\Longrightarrow z-z_1 = z'-z_1\Longrightarrow z=z'$$

 Para provar que $\mathbb{C}\setminus L(U)$ tem interior não vazio, considere $w=L(z)\in L(U)$ para algum $z\in U$. Observe que $w+2\pi i\notin L(U)$. De fato, se $w+2\pi i \in L(U)$, então teríamos que $w+2\pi i = L(z')$ para algum $z'\in U$, o que implicaria que: $$z'-z_1 = \exp({L(w+2\pi i)}) = \exp({L(w)}) = z - z_1\Longrightarrow z'=z$$
 E daí, $$L(z')=L(z)\Longrightarrow w = w+2\pi i$$o que é uma contradição. Assim, se $D_1$ é um disco tal que $D_1\subset L(U)$, então $D_1+2\pi i = \{z+2\pi i\,:\, z\in D_1\}\subset \mathbb{C}\setminus L(U)$, o que prova a afirmação (1). 

 Agora, considere um disco $D_r(w_0)\subset(\mathbb{C}\setminus L(U))$, para algum $r>0$. A transformação de Moebius dada por $h(w) = \dfrac{r}{w-w_0}$ é tal que $h(\hat{\mathbb{C}}\setminus D_r(w_0)) = D$. Como $L(U)\subset(\hat{\mathbb{C}}\setminus D_r(W_0))$, temos que $h(L(U))\subset h(\hat{\mathbb{C}}\setminus D_r(w_0)) = D$, e portanto, $h(L(U))\subset D$.
 
Por outro lado, se $\xi_0 = h(L(z_0))$ para algum $z_0\in U$, então a transformação de Moebius $$g(\xi)=\dfrac{\xi -\xi_0}{1-\overline{\xi_0}\xi}$$ é tal que $g(D) = D$ e $g(\xi_0)=0$.

Agora, basta tomar $f_1=g\circ h\circ L$. Por construção, $f_1$ é injetiva e $f_1(z_0) = g(h(L(z_0)) = g(\xi_0) = 0$ e $f_1(U) = g(h(L(U)))\subset D$. Podemos supor, além disso, que existe $\rho>0$ tal que $f'(z_0)=\rho\cdot\exp{(i\theta)}$. Daí, basta tomarmos $f(z)=\exp{-i\theta}f_1(z)$ e teremos que $f\in\mathcal{F}$, donde $\mathcal{F}\neq\emptyset$.

Agora, vamos mostrar que existe $f\in\mathcal{F}$ tal que $f(U) = D$. Observe primeiro que como $|f(z)|<1,\,\forall z\in U,\,\forall f\in\mathcal{F}$, a família $\mathcal{F}$ é localmente uniformemente limitada.

Pelo Teorema de Montel, $\mathcal{F}$ é uma família normal de funções, isto é, o seu fecho em $\mathcal{H}(U)$ é compacto.

Afirmação (2): $\overline{\mathcal{F}} = \mathcal{F}\cup \{0\}$, onde $0(z)=0, \forall z\in U$. 

De fato, sabemos que se uma sequência de funções injetivas $(f_n)_{n\geq 1}$ em $\mathcal{H}(U)$ converge uniformemente nas partes compactas para uma função $f$, então $f$ é injetiva ou constante. Por definição, se tomarmos $f\in\overline{\mathcal{F}}\setminus\mathcal{F}$, existe uma sequência $(f_n)_{n\geq 1}\in\mathcal{F}$ tal que $f\longrightarrow f$ uniformemente nas partes compactas. Veja que se $f$ for constante, então claramente $f\equiv 0$, pois $f_n(z_0)=0,\forall n\in\mathbb{N}$. Por outro lado, se $f$ não for constante, então $f$ é injetiva e além disso, $f(z_0)=\displaystyle\lim_{n\to\infty}f_n(z_0)=0$ e $f'(z_0)=\displaystyle\lim_{n\to\infty}f'(z_0)\geq 0$. Mas não podemos ter $f'(z_0)=0$ pois $f$ é injetiva. Logo, $f'(z_0)\in(0,+\infty)$. Para concluirmos a afirmação (2), basta mostrarmos que $f(U)\subset D$. De fato, se $f(U)\cap(\mathbb{C}\setminus D)\neq\emptyset$, então existe $z_1\in U$ tal que $f(z_1)=w_1$, com $|w_1|\geq 1$. 

Existe um Corolário do Teorema de Hurwitz que afirma o seguinte: Seja $(f_n)_{n\geq 1}$ uma sequência em $\mathcal{M}(U)$ tal que $f_n\longrightarrow f\in\mathcal{M}(U)$ uniformemente nas partes compactas. Suponha que $z_0\in U$ seja um polo (ou zero) de ordem $k\geq 1$ de $f$. Para todo $r>0$ suficientemente pequeno, existe $n=n(r)\geq 1$ tal que se $n\geq n_0$ então $f_n$ possui $k$ polos (ou zeros) contados com multiplicidade em $D_r(z_0)$. 

Pelo Corolário acima, existiria $n_0\geq 1$ tal que se $n\geq n_0$ então a equação $f_n(z)=w_1$ possuiria pelo menos uma solução em $U$, o que implicaria que $f_n(U)\cap(\mathbb{C\setminus D})\neq\emptyset$, e daí que $f_n\notin\mathcal{F}$, o que é uma contradição. Portanto, $\overline{\mathcal{F}}=\mathcal{F}\cup\{0\}$.

Agora, defina $$\begin{array}{cccc}\varphi\colon & \overline{\mathcal{F}} & \longrightarrow & [0,+\infty) \\ & f & \longmapsto & f'(z_0)\end{array}$$ É claro que $\varphi$ é contínua, pois se $(f_n)_{n\geq 1}$ é uma sequência em $\overline{\mathcal{F}}$ tal que $f_n\longrightarrow f$ uniformemente nas partes compactas, então $f'_n\longrightarrow f'$, logo, $$\displaystyle\lim_{n\to\infty}\varphi(f_n) = \displaystyle\lim_{n\to\infty}f'(z_0) = f'(z_0) = \varphi(f)$$ Como $\overline{\mathcal{F}}$ é compacto, sua imagem por $\varphi$, $\varphi(\overline{\mathcal{F}})$, é um conjunto compacto também. Logo, existe $f_0\in\overline{\mathcal{F}}$ tal que $$\varphi(f_0) = sup{\{\varphi(f)\,:\,f\in\overline{\mathcal{F}}\}}$$ Note que $f_0$ não pode ser constante, pois $\mathcal{F}\neq\emptyset$ e portanto $f'(z_0)>0$. Assim, $f_0\in\mathcal{F}$.

Afirmação (3): $f_0(U)=D$.

A ideia será a seguinte: se $f\in\mathcal{F}$ é tal que $f(U)\neq D$, então vamos construir uma aplicação $g\in\mathcal{F}$ tal que $g'(z_0)>f'(z_0)$, o que será um absurdo.

Para isso, fixemo então $f\in\mathcal{F}$ tal que $f(U)\neq D$, digamos $f(U)=V\subset D$ e considere $w_0\in D\setminus V$. Como $f\colon U\longrightarrow D$ é um homeomorfismo, então $f(z_0)\in V$ e $V$ é simplesmente conexo. Afirmamos então que existe $h\colon V\longrightarrow D$ holomorfa e injetiva tal que $h(0) = 0$ e $h'(0)\in(1,+\infty)$. De fato, considere a transformação de Moebius: $$T(z)=\dfrac{z-w_0}{1-\overline{w_0}z}$$

Como $T(D) = D$ e $V\subset D$, é claro que $T(V)\subset D$. Por outro lado, como $T(V)$ é simplesmente conexo, e $0 = T(w_0)\notin T(V)$, podemos definir um ramo do logaritmo em $T(V)$, digamos $L$. Seja $h_1(z) = \exp{\left(\dfrac{1}{2}L(T(z))\right)}$. Observe que $h_1$ é um ramo da raiz quadrada de $T(z)$ em $V$, pois: $$(h_1(z))^2 = \left(\exp{\left(\dfrac{1}{2}L(T(z))\right)}\right)^2 = \exp{(L(T(z)))} = T(z)$$ Fica evidente que $h_1$ é injetiva, pois para todos $z_1,z_2\in V$ tais que, $$(h_1(z))^2 = (h_1(z_2))^2\Longrightarrow T(z_1)=T(z_2)\Longrightarrow z_1=z_2$$
Note também que se $z\in V$, então $|(h_1(z))|^2 = |T(z)| < 1 $, donde $|h_1(z)|<1$, e por isso $h_1(V)\subset D$. Assim, tomemos então $h=S\circ h_1$, onde $$S(w) = \lambda\dfrac{w-h_1(0)}{1-\overline{h_1(0)}w}$$ onde $\lambda=-\dfrac{|h_1(0)}{h_1(0)}$. Como $|\lambda| = 1$, temos que $h(V) = (S\circ h_1)(V)\subset S(D) = D$. Além disso, $h$ é injetiva por construção e $$h(0) = S(h_1(0)) = \lambda\dfrac{h_1(0)-h_1(0)}{1-\overline{h_1(0)}h_1(0)} = 0$$ Por outro lado, $$h'(0) = S'(h_1(0))\cdot h'_1(0) = \dfrac{\lambda}{1-|h_1(0)|^2}\cdot\dfrac{1}{2}\cdot h_1(0)\cdot\dfrac{T'(0)}{T(0)} = \dfrac{1+|w_0|}{2\sqrt{|w_0|}}$$

Como $|w_0|<1$, vem que $1+|w_0| > 2\sqrt{|w_0|}$ e daí, $h'(0) > 1$, como queríamos.

Finalmente, obtemos $g$ da seguinte forma: $g=h\circ f$. Novamente fica evidente que $g\in\mathcal{F}$ e além disso, $$g'(z_0) = h'(f(z_0))\cdot f'(z_0) = h'(0)\cdot f'(z_0) > f'(z_0)$$ isto é, $g'(z_0)> f'(z_0) = $ pois $h'(0)>1$, o que é uma contradição, pois obtemos que $$g'(z_0) > \varphi(f_0) = sup{\{\varphi(f)\,:\,f\in\overline{\mathcal{F}}\}}$$
Portanto, $f_0(U) = D$, como queríamos.

Resta agora provarmos a unicidade.

De fato, suponhamos que $f_0,f_1\in\mathcal{F}$ são tais que $f_0(U) = f_1(U) = D$. Podemos definir então a aplicação: $$T = f_1\circ f_0^{-1}\colon D\longrightarrow D$$
que é um automorfismo holomorfo em $D$. Além disso, temos que: $$T(0) = (f_1\circ f_0^{-1})(0) = f_1(f_0^{-1}(0)) = f_1(z_0) = 0$$ E que: $$T'(0) = (f_1\circ f_0^{-1})'(0) = f'_1(f_0^{-1}(0))\cdot (f_0^{-1})'(0)=\dfrac{f'_1(z_0)}{f'_0(f_0^{-1}(0))} = \dfrac{f'_1(z_0)}{f'_0(z_0)}>0$$

Como todo automorfismo holomorfo de $D$ é da forma $S(z) = \lambda\dfrac{z-z_0}{1-\overline{z_0}z}$ com $|\lambda|=1$ e $|z_0|<1$, temos então que: $$T(0) = S(0) = \lambda(-z_0) = 0$$ donde $$|T(0)| = |\lambda||z_0|=|z_0| = 0\Longrightarrow z_0 = 0$$ e logo, $T(z) = z$ é a transformação identidade. Portanto, $f_0 = f_1$, e segue a unicidade, como queríamos.
Isto finaliza a prova o Teorema de Riemann.


\end{proof}

\subsection{Lívia}
\begin{proof}[\Cref{p1}]

Queremos um biholomorfismo $ \varphi: H^{+} \to D(0,1)\backslash[x,1)$. Para isso, podemos usar \textit{Transformações de Cayley}, que mapeiam o plano superior para o disco unitário. Mais precisamente, tal transformação mapeia a reta real - $ \R \cup \{\infty\} $ - para o círculo unitário - $ C(0,1) = \partial D(0,1) $ - e o eixo imaginário para o intervalo real $ (-1,1) \subset D(0,1) $. Sua forma geral pode ser expressa por:

\begin{align*}
    g: H^{+} \to D(0,1) \implies g(z) = \frac{z-i}{z+i}
\end{align*}

Entretanto, como queremos excluir o intervalo real $[x,1)$ do disco unitário, com $x \in (-1,1)$, devemos primeiro alterar o intervalo do eixo imaginário de $H^{+}$. Para isso, basta encontrar um biholomorfismo $H^{+} \to H^{+}\backslash\{it: t \geq \alpha\}$, onde $\alpha$ é expresso por:

\begin{align*}
    g(i\alpha) = x \implies \alpha = \frac{1+x}{1-x}
\end{align*}

Passo 1) $H^{+}\backslash\{it: t \geq \alpha\} \to \C\backslash\{(-\infty,\alpha^2] \cup [0,\infty)\}$ 
\[ z \to z^2 \]

Passo 2) $\C\backslash\{(-\infty,\alpha^2] \cup [0,\infty)\} \to \C\backslash[0,\infty)$
\[ z^2 \to \frac{z^2}{z^2+\alpha^2} \]

Passo 3) $\C\backslash[0,\infty) \to H^{+}$
\[ \frac{z^2}{z^2+\alpha^2} \to \sqrt{\frac{z^2}{z^2+\alpha^2}} \]

Logo, podemos denotar a composição desses passos como a função $f$, que é uma função holomorfa, pois é composição de função elementares - por isso, holomorfas. Já que $f(z) \neq 0, \forall z \in H^{+}\backslash\{it: t\geq \alpha\}$, ela admite inversa. Assim, $f$ é o primeiro biholomorfismo procurado. Com isso, temos:

\begin{align*}
    \sqrt{\frac{z^2}{z^2+\alpha^2}} = w \implies f^{-1}(w)=\frac{w\alpha}{\sqrt{1-w^2}}
\end{align*}

onde $f^{-1}: H^{+} \to H^{+}\backslash\{it: t\geq\alpha\}$.

Agora, basta compor $f^{-1}$ e $g$, do que segue:

\[ \varphi: H^{+} \to D(0,1)\backslash[x,1) \]
\[ z \to  \frac{w\alpha-i\sqrt{1-w^2}}{w\alpha+i\sqrt{1-w^2}}\]

Sabemos que toda Transformação de Cayley é meromorfa, mas como nesse domínio ela não admite polos, podemos afirmar que $g$ é holomorfa. Com isso, vemos que $\varphi$ é uma composição de funções holomorfas e, portanto, holomorfa. De modo análogo, concluimos o mesmo sobre sua inversa. Logo, $\varphi$ é o biholomorfismo procurado entre $H^{+}$ e $D(0,1)\backslash[x,1)$.

\end{proof}



\end{document}
